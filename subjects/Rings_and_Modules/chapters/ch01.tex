\chapter{RINGS}

\textit{January 19th.}

Of course, we begin with the definition of a ring.

\begin{definition}
    A \eax{ring} is a triple $(R, +, \cdot)$ where $R$ is a set, and $+$ and $\cdot$ are binary operations on $R$ such that the following axioms are satisfied:
    \begin{itemize}
        \item $(R, +)$ is an abelian group. The identity element of this group is denoted by $0_{R}$, and the (additive) inverse of an element $a \in R$ is denoted by $-a$.
        \item The property of \eax{associativity} of $\cdot$ holds; i.e., for all $a, b, c \in R$, we have $(a \cdot b) \cdot c = a \cdot (b \cdot c)$.
        \item The property of \eax{distributivity} of $\cdot$ over $+$ holds; i.e., for all $a, b, c \in R$, we have
        \begin{align}
            a \cdot (b + c) &= a \cdot b + a \cdot c, \\
            (a + b) \cdot c &= a \cdot c + b \cdot c.
        \end{align}
    \end{itemize}
\end{definition}

Rings may be written simply as $R$ instead of the triple. The ring $R$ is termed a \eax{ring with unity} if there exists an element $1_{R} \in R$ such that for all $a \in R$, we have $1_{R} \cdot a = a \cdot 1_{R} = a$. Some examples of rings with unity include $\Z$, $\Q$, $\R$, $\C$, $M_{n}(\R)$ with the usual addition and multiplication. A ring $R$ is said to be a \eax{commutative ring} if for all $a, b \in R$, we have $a \cdot b = b \cdot a$. Examples of commutative rings include $\Z$, $\Q$, $\R$, $\C$, but $M_{n}(\R)$ is not commutative for $n \geq 2$. Lastly, a commutative ring $R$ with unity is termed a \eax{field} if every non-zero element of $R$ has a multiplicative inverse; i.e., for every $a \in R \setminus \{0_{R}\}$, there exists an element $b \in R$ such that $a \cdot b = b \cdot a = 1_{R}$. Examples of fields include $\Q$, $\R$, $\C$, but $\Z$ is not a field.

Example of rings without unity include $2\Z$ with the usual addition and multiplication, and the set of all continuous functions from $\R$ to $\R$ that vanish at $0$, with the usual addition and multiplication of functions. Another class of rings we previously studied was $\Z/n\Z$ for $n \geq 2$, with the usual addition and multiplication modulo $n$. This ring has unity, but is a field if and only if $n$ is prime.

\begin{definition}
    Let $R$ be a ring with unity. An element $a \in R$ is called a \eax{unit} if there exists an element $b \in R$ such that $a \cdot b = b \cdot a = 1_{R}$.
\end{definition}

For example, in the ring $\Z/n\Z$, an element $\bar{a}$ is a unit if and only if $\gcd(a, n) = 1$. The set of all units in a ring $R$ with unity is denoted by $R^{\times}$. It can be easily verified that $(R^{\times}, \cdot)$ is an abelian group.

\section{Properties and Maps}
Some basic properties may be inferred.

\begin{proposition}
    Let $R$ be a ring with unity. Then,
    \begin{itemize}
    \item $1_{R}$ is the unique multiplicative identity in $R$.
    \item $1_{R} \cdot 0_{R} = 0_{R}$. In general, $a \cdot 0_{R} = 0_{R}$ for all $a \in R$.
    \item $-1_{R} \cdot a = -a$ for all $a \in R$.
\end{itemize}
\begin{proof}
    \begin{itemize}
        \item This is left as an exercise to the reader.
        \item $1_{R} \cdot 0_{R} = 1_{R}$ is trivial since $1_{R}$ is the multiplicative identity. For the general case, let $a \in R$. Then,
        \begin{align}
            a \cdot 0_{R} = a \cdot (0_{R} + 0_{R}) = a \cdot 0_{R} + a \cdot 0_{R} \implies a \cdot 0_{R} = 0_{R}
        \end{align}
        by the addition of $-(a \cdot 0_{R})$ on both sides.
        \item Let $a \in R$. Then,
        \begin{align}
            (-1_{R} \cdot a) + a = (-1_{R} + 1_{R}) \cdot a = 0_{R} \implies -1_{R} \cdot a = -a.
        \end{align}
    \end{itemize}
\end{proof}
\end{proposition}

The subscript $R$ in $0_{R}$ and $1_{R}$ may be dropped when the context is clear. We move on to some special maps.

\begin{definition}
    A \eax{ring homomorphism} is a map $\varphi : (R, +, \cdot) \to (S, \oplus, \odot)$ between two rings such that for all $a, b \in R$, we have
    \begin{align}
        \varphi(a + b) = \varphi(a) \oplus \varphi(b), \quad \varphi(a \cdot b) = \varphi(a) \odot \varphi(b).
    \end{align}
\end{definition}
Most of the time, we shall drop $\oplus$ and $\odot$ when the context is clear. Some examples of ring homomorphisms include the map $\varphi : \Z \to \Z/n\Z$ defined by $\varphi(a) = \bar{a}$ for all $a \in \Z$, and the inclusion map from $\Z$ to $\Q$. Non-examples include $n \mapsto -n$ from $\Z$ to $\Z$, and the determinant map from $M_{n}(\R)$ to $\R$.

Let $(\Z \times \Z, +, \cdot)$ be the ring where addition and multiplication are defined component-wise. Then the map $Z \to \Z \times \Z$ defined by $a \mapsto (a, 0)$ is a ring homomorphism since it preserves both addition and multiplication. However, the unity of $\Z$ is mapped to $(1, 0)$, which is not the unity of $\Z \times \Z$. Thus, ring homomorphisms need not map unity to unity.

\begin{definition}
    Let $R$ be a ring with $S \subseteq R$ a subset. Then, $S$ is called a \eax{subring} of $R$ if $(S, +, \cdot)$ is itself a ring with the operations inherited from $R$.
\end{definition}
Again, even if $R$ has unity, a subring $S$ need not have the same unity as $R$ or even a unity at all.
\\ \\
\textit{January 23rd.}

\begin{definition}
    A ring homomorphism $\varphi: R \to S$ is termed a \eax{ring monomorphism} if it is injective, a \eax{ring epimorphism} if it is surjective, and a \eax{ring isomorphism} if it is bijective. If there exists a ring isomorphism from $R$ to $S$, then $R$ and $S$ are said to be \eax{isomorphic}, denoted by $R \cong S$.
\end{definition}

Note that if $\varphi: R \to S$ is bijective, then its inverse $\varphi^{-1} : S \to R$ is a ring homomorphism. We look at some examples of rings and mappings.

\begin{example}
    Let $X$ be any set and let $R \defeq \{f:X \to \R\}$ be the set of all functions from $X$ to $\R$. Then, $(R, +, \cdot)$ is a ring where addition and multiplication are defined pointwise; i.e., for all $f, g \in R$ and $x \in X$, $(f + g)(x) \defeq f(x) + g(x)$ and $(f \cdot g)(x) \defeq f(x) \cdot g(x)$. The additive identity is the zero function $0:X \to \R$ defined by $0(x) = 0$ for all $x \in X$, and the multiplicative identity is the constant function $1:X \to \R$ defined by $1(x) = 1$ for all $x \in X$. It is easy to verify that all ring axioms are satisfied. Moreover, this ring is commutative and has unity. Note that $\R$ can be replaced by any ring $S$ to form the ring of functions from $X$ to $S$. In such a case, $R$ is a (commutative) ring with unity if and only if $S$ is a (commutative) ring with unity.

    In the special case that $X = \{1,2,\ldots,n\}$ for some $n \in \N$, the ring $R$ is isomorphic to the ring $(\R^{n},+,\cdot)$ where addition and multiplication are defined component-wise. The isomorphism $\varphi : R \to \R^{n}$ is given by $\varphi(f) = (f(1), f(2), \ldots, f(n))$ for all $f \in R$.
\end{example}

\begin{example}
    Continuing from the previous example, let $X = [a,b]$. Note that the $R$ in this case is the set of all functions from the interval $[a,b]$ to $\R$, which is not a very manageable set. Thus, we may consider the subset $C([a,b], \R) \subseteq R$ consisting of all continuous functions from $[a,b]$ to $\R$. It is easy to verify that $C([a,b], \R)$ is a subring of $R$. Similarly, one defined $C^{n}([a,b], \R)$ to be the set of all $n$-times continuously differentiable functions from $[a,b]$ to $\R$, and $C^{\infty}([a,b], \R)$ to be the set of all infinitely differentiable functions from $[a,b]$ to $\R$. Both of these are subrings of $R$ as well.
\end{example}

\begin{example}
    The set $\Z[i] \defeq \{a + bi : a, b \in \Z\}$ is a subring of the field $\C$. It is easy to verify that $\Z[i]$ is a ring with unity, but it is not a field since, for example, the element $1 + i$ does not have a multiplicative inverse in $\Z[i]$. Note that there is a natural bijection $\varphi : \Z[i] \to \Z^{2}$ defined by $\varphi(a + bi) = (a, b)$ for all $a + bi \in \Z[i]$, where $\Z^{2}$ has component-wise addition and multiplication. However, this map is not a ring isomorphism since it does not preserve multiplication; for example, $\varphi(i \cdot i) = \varphi(-1) = (-1, 0)$, but $\varphi(i) \cdot \varphi(i) = (0, 1) \cdot (0, 1) = (0, 1)$.
\end{example}

\subsection{Polynomials}

Let $R$ be a ring. The polynomial ring in the variable $x$ with coefficients from $R$ is defined as follows:
\begin{definition}
    The \eax{polynomial ring} $R[x]$ is defined as
    \begin{align}
        R[x] \defeq \{f:\N_{0} \to R \mid f(n) = 0 \text{ for all but finitely many } n \in \N_{0}\}.
    \end{align}
    The elements of $R[x]$ are called \eax{polynomials} in the variable $x$ with coefficients from $R$. For $f, g \in R[x]$ and $n \in \N_{0}$, addition is defined as
    \begin{align}
        (f + g)(n) \defeq f(n) + g(n) \quad \text{ for all } n \in \N_{0},
    \end{align}
    and multiplication is defined as
    \begin{align}
        (f \cdot g)(n) \defeq \sum_{k=0}^{n} f(k) \cdot g(n-k) \quad \text{ for all } n \in \N_{0}.
    \end{align}
\end{definition}

Alternatively, a polynomial $f \in R[x]$ may be expressed in the form
\begin{align}
    f(x) = a_{0} + a_{1}x + a_{2}x^{2} + \cdots + a_{n}x^{n},
\end{align}
where $a_{i} = f(i)$ for all $0 \leq i \leq n$ and $f(k) = 0$ for all $k > n$. For $0 \neq f \in R[x]$ as above with $a_{n} \neq 0_{R}$, the integer $n$ is called the \eax{degree} of $f$, denoted by $\deg(f)$. The degree of the zero polynomial is usually left undefined, or changed upon convention. Also note that $f \cdot g \in R[x]$ since $f \cdot g(k) = 0_{R}$ for all $k > \deg(f) + \deg(g)$.

\begin{proposition}
    For a ring $R$, the polynomial ring $R[x]$ is, indeed, a ring with unity under the operations defined above. If $R$ is commutative, then so is $R[x]$. The map $\iota : R \to R[x]$ defined by $\iota(a) = f_{a}$ where $f_{a}(0) = a$ and $f_{a}(n) = 0_{R}$ for all $n \geq 1$ is a ring monomorphism.
\end{proposition}

\begin{proof}
    That $(R[x],+)$ forms an abelian group is clear. The associativity of multiplication is verified as follows: let $f, g, h \in R[x]$ and $n \in \N_{0}$. Then,
    \begin{align}
        ((f \cdot g) \cdot h)(n) &=  \sum_{k=0}^{n} (f \cdot g)(k) \cdot h(n-k) = \sum_{k=0}^{n} \left( \sum_{j=0}^{k} f(j) \cdot g(k-j) \right) \cdot h(n-k) \notag \\ &= \sum_{j=0}^{n} f(j) \cdot \left( \sum_{k=j}^{n} g(k-j) \cdot h(n-k) \right) = \sum_{j=0}^{n} f(j) \cdot (g \cdot h)(n-j) = (f \cdot (g \cdot h))(n).
    \end{align}
    The distributive properties follow similarly. The unity in $R[x]$ is the polynomial $1_{R[x]}$ defined by $1_{R[x]}(0) = 1_{R}$ and $1_{R[x]}(n) = 0_{R}$ for all $n \geq 1$. Finally, it is easy to verify that $\iota$ is a ring homomorphism, and it is injective since $\iota(a) = \iota(b)$ implies that $a = b$.
\end{proof}

With $R[x]$ established as a ring, we may consider a higher level of abstraction, by considering polynomials over this polynomial ring itself; that is, $(R[x])[y]$. Elements of this ring look like
\begin{align}
    f(x,y) = a_{00} + a_{10}x + a_{01}y + a_{20}x^{2} + a_{11}xy + a_{02}y^{2} + \cdots + a_{mn}x^{m}y^{n},
\end{align}
where $a_{ij} \in R$ for all $i, j \geq 0$ and $a_{ij} = 0_{R}$ for all but finitely many pairs $(i,j)$. We have already shown that $R[x]$ is a ring, so it follows that $(R[x])[y]$ is also a ring. This ring is usually denoted by $R[x,y]$. For $f \in R[x,y]$ as above with $a_{mn} \neq 0_{R}$, the degree of $f$ is defined as $\deg(f) = m + n$. Similarly, one may define $R[x_{1}, x_{2}, \ldots, x_{n}]$ for any $n \in \N$. For a countable number of indeterminates, one may define $R[x_{1}, x_{2}, x_{3}, \ldots]$ as the union $\bigcup_{n=1}^{\infty} R[x_{1}, x_{2}, \ldots, x_{n}]$.

\begin{example}
    Let $e \in \R$ be the Euler's number (or any transcendental number). Then $\Z[e] \subseteq \C$ is the smallest subring of $\C$ containing both $\Z$ and $e$. Here, $\Z[e]$ consists of all polynomials in $e$ with integer coefficients; i.e., all elements of the form $a_{0} + a_{1}e + a_{2}e^{2} + \cdots + a_{n}e^{n}$ where $n \geq 0$ and $a_{i} \in \Z$. Since $e$ is transcendental, there are no non-trivial polynomial relations among the powers of $e$ with integer coefficients. Thus, the map $\varphi : \Z[x] \to \Z[e]$ defined by $\varphi(f) = f(e)$ for all $f \in \Z[x]$ is a ring isomorphism.
\end{example}

\section{Ideals}

\begin{definition}
    Let $R$ be a commutative ring with unity. A subset $I \subseteq R$ is called an \eax{ideal} of $R$ if the following conditions hold:
    \begin{itemize}
        \item for all $a, b \in I$, we have $a + b \in I$,
        \item for all $a \in I$ and $r \in R$, we have $r \cdot a \in I$.
    \end{itemize}
\end{definition}

Note that the first condition implies that $(I, +)$ is a subgroup of $(R, +)$. Some examples of ideals include the set $\{0_{R}\}$, the ring $R$ itself, and the set $n\Z = \{nk : k \in \Z\}$ for any $n \in \Z_{\geq 0}$ as an ideal of the ring $\Z$. A non-example is $\Z$ in $\R$; it is a subring, but not an ideal since, for example, $1 \in \Z$ but $\pi \cdot 1 = \pi \notin \Z$. Note that if $1_{R} \in I$, then $I = R$.

\begin{example}
    Let us look at ideals of $\R$. Trivially, $\{0\}$ and $\R$ are ideals of $\R$. We claim that these are the only ideals of $\R$. To see this, let $I$ be any ideal of $\R$ such that $I \neq \{0\}$. Then, there exists some $a \in I$ such that $a \neq 0$. Since $\R$ is a field, $a$ has a multiplicative inverse $a^{-1} \in \R$. Thus, $1 = a^{-1} \cdot a \in I$, which implies that $I = \R$. In fact, this argument shows that in any field, the only ideals are the zero ideal and the field itself.
\end{example}

\begin{example}
    We examine ideals of $\Z$. From group theory, we know that every subgroup of $(\Z, +)$ is of the form $n\Z$ for some $n \in \Z_{\geq 0}$, so $n\Z$ are the only candidates for ideals of $\Z$. In fact, each $n\Z$ is an ideal of $\Z$ since for all $a, b \in n\Z$, we have $a + b \in n\Z$, and for all $a \in n\Z$ and $r \in \Z$, we have $r \cdot a \in n\Z$. Thus, the ideals of $\Z$ are precisely the sets $n\Z$ for $n \in \Z_{\geq 0}$, and $\Z$.
\end{example}

\begin{proposition}
    Let $f:R \to S$ be a ring homomorphism between two commutative rings with unity. Then, the \eax{kernel} of $f$, defined as
    \begin{align}
        \ker f \defeq \{a \in R \mid f(a) = 0_{S}\},
    \end{align}
    is an ideal of $R$. Moreover, $f$ is a ring monomorphism if and only if $\ker f = \{0_{R}\}$.
\end{proposition}
\begin{proof}
    Let $a, b \in \ker f$ and $r \in R$. Then,
    \begin{align}
        f(a + b) = f(a) + f(b) = 0_{S} + 0_{S} = 0_{S},
    \end{align}
    so $a + b \in \ker f$. Also,
    \begin{align}
        f(r \cdot a) = f(r) \cdot f(a) = f(r) \cdot 0_{S} = 0_{S},
    \end{align}
    so $r \cdot a \in \ker f$. Thus, $\ker f$ is an ideal of $R$.

    Now, suppose that $f$ is a ring monomorphism. Let $a \in \ker f$. Then, $f(a) = 0_{S} = f(0_{R})$. Since $f$ is injective, we have $a = 0_{R}$, so $\ker f = \{0_{R}\}$. Conversely, suppose that $\ker f = \{0_{R}\}$. Let $a, b \in R$ such that $f(a) = f(b)$. Then,
    \begin{align}
        f(a - b) = f(a) - f(b) = 0_{S},
    \end{align}
    so $a - b \in \ker f$. Thus, $a - b = 0_{R}$, which implies that $a = b$. Therefore, $f$ is injective.
\end{proof}

\noindent \textit{January 24th.}

Let $R$ be a ring with unity and $R_{i}$ be a colelction of subrings of $R$ containing the nuity. Then $\bigcap_{i} R_{i}$ is also a subring of $R$ containing the unity. If $I_{j}$ is a collection of ideals of $R$, then $\bigcap_{j} I_{j}$ is also an ideal of $R$. Thus, given any subset $S \subseteq R$, we may define the ideal generated.

\begin{definition}
    Let $R$ be a commutative ring with unity and $I \subseteq R$ be an ideal. Let $S \subseteq I$ be a set. We say $S$ is a \eax{generating set} of $I$ if $I$ is the smallest ideal containing $S$.
\end{definition}

\begin{proposition}
    Let $R$ be a commutative ring with unity and $S \subseteq R$ be any subset. Then, the ideal generated by $S$, denoted by $(S)$, is given by
    \begin{align}
        (S) = \left\{ \sum_{i=1}^{n} r_{i} s_{i} : n \geq 0, r_{i} \in R, s_{i} \in S \text{ for all } 1 \leq i \leq n \right\}.
    \end{align}
\end{proposition}

\begin{proof}
    Let $S \subseteq I$, a subset of an ideal. We claim that $(S) \subseteq I$. Let $\alpha \in I$. THen, $\alpha = r_{1}x_{1} + \cdots + r_{n}x_{n}$ for some $n \geq 0$, $r_{i} \in R$ and $x_{i} \in S$ for all $1 \leq i \leq n$. Since $I$ is an ideal, we have $r_{i}x_{i} \in I$ for all $1 \leq i \leq n$, and thus $\alpha \in I$. Therefore, $(S) \subseteq I$.
\end{proof}

With this, we introduce the notation that if $\{x_{1},\ldots,x_{n}\} \subseteq R$, then $I = (x_{1},\ldots,x_{n}) = Rx_{1} + \cdots + Rx_{n}$ is the ideal generated by $x_{1},\ldots,x_{n}$. Let us look at some exmaples.

\begin{example}
    In the ring $\Z$, $(2,3) = \Z$ since $1 = 3 - 1 \cdot 2 \in (2,3)$. More generally, for any $a, b \in \Z$, we have $(a,b) = \Z$ if and only if $\gcd(a,b) = 1$. Moreover, in $\Z$, every ideal can be generated by a single element; i.e., every ideal is of the form $(n)$ for some $n \in \Z_{\geq 0}$.
\end{example}

\begin{example}
    In $\Z[x]$, the ideal $(2, x)$ consists of all polynomials with integer coefficients where the constant term is even. That is, $(2,x) = 2\Z[x] + x\Z[x]$. 
\end{example}

Also note that a union of ideals need not be an ideal. For example, in $\Z$, the sets $2\Z$ and $3\Z$ are ideals, but their union $2\Z \cup 3\Z$ is not an ideal. This, however, calls for a more general construction.

\begin{definition}
    Let $R$ be a commutative ring with unity. If $I_{1}, I_{2}$ are two ideals, we then define their sum as $I_{1} + I_{2} = (I_{1} \cup I_{2})$.
\end{definition}
It is easy to verify that $I_{1} + I_{2} = \{a + b \mid a \in I_{1}, b \in I_{2}\}$. This definition may be extended to a finite number of ideals in the obvious way.

\section{Other Rings}

\begin{definition}
    Let $G$ be a group and $k$ be a field. We define $R[G]$ to be the set of all functions $f:G \to k$ such that $f(x) = 0$ for all but finitely many $x \in G$. Addition is defined pointwise as
    \begin{align}
        (f + g)(x) = f(x) + g(x) \quad \text{ for all } x \in G,
    \end{align}
    and multiplication is defined as
    \begin{align}
        (f \cdot g)(x) = \sum_{y z = x} f(y) g(z) = \sum_{y \in G} f(xy^{-1}) g(y) \quad \text{ for all } x \in G.
    \end{align}
    The ring $R[G]$ is called the \eax{group ring} of $G$ over $k$.
\end{definition}

If $G$ is a finite group with $G = \{e,x_{2},\ldots,x_{n}\}$ then $R[G] = \{a_{1}e + a_{2}x_{2} + \cdots + a_{n}x_{n} \mid a_{i} \in \C\}$. Verify that $R[G]$ is a ring with unity under the operations defined above. This ring, however, may not be commutative.

\begin{definition}
    Let $R$ be a ring and $x \in R$. $x$ is termed \eax{nilpotent} if there exists some $n \in \N$ such that $x^{n} = 0$. If $R$ is commutative, $x \in R$ is called a \eax{zero divisor} if there exists some $y \in R \setminus \{0\}$ such that $x \cdot y = 0$.
\end{definition}

Note that nilpotents are zero divisors in a commutative ring, but the converse need not be true. For example, in the ring $\Z/6\Z$, the element $\bar{2}$ is a zero divisor since $\bar{2} \cdot \bar{3} = \bar{0}$, but it is not nilpotent since $\bar{2}^{n} \neq \bar{0}$ for all $n \geq 1$.

\begin{definition}
    A commutative ring with unity $R$ is called a \eax{reduced ring} if it has no non-zero nilpotent elements. It is called an \eax{integral domain} if it has no non-zero zero divisors.
\end{definition}

\begin{proposition}
    Let $R$ be an integral domain. Then, if $x,y \in R$ are such that $x \cdot y = 0$, then either $x = 0$ or $y = 0$.
\end{proposition}
\begin{proof}
    If $x \neq 0$, then since $R$ is an integral domain, $x$ is not a zero divisor. Thus, $y$ must be $0$. Similarly, if $y \neq 0$, then $x$ must be $0$.
\end{proof}

\begin{proposition}
    Every integral domain is a reduced ring.
\end{proposition}
\begin{proof}
    Let $R$ be an integral domain and let $x \in R$ be nilpotent. Then, there exists some $n \in \N$ such that $x^{n} = 0$. If $x \neq 0$, then since $R$ is an integral domain, $x$ is not a zero divisor. However, this contradicts the fact that $x^{n} = 0$. Thus, we must have $x = 0$, so $R$ has no non-zero nilpotent elements.
\end{proof}

\noindent \textit{January 29th.}

In an integral domain $R$, if $ab = ac$ for some $a, b, c \in R$, then either $a = 0$ or $b = c$. Let us look at some examples of integral domains.

\begin{example}
    The ring $\Z$ is an integral domain since it has no non-zero zero divisors. Similarly, the rings $\Q$, $\R$, and $\C$ are integral domains as well. More generally, any field is an integral domain. Moreover, $\Z/n\Z$ is an integral domain if and only if $n$ is prime.
\end{example}

\begin{example}
    Note that $R$ is an integral domain if and only if $R[x]$ is an integral domain. 
\end{example}

Another small result is as follows: if $R$ is an integral domain and $R'$ is a subring of $R$ containing the unity, then $R'$ is also an integral domain. Some non-examples of integral domains include $\Z^{2}$, $C[0,1]$, $C^{\infty}[0,1]$, etc.



\section{Quotient Rings and Isomorphism Theorems}

From here on, we shall assume that all rings are commutative with unity unless otherwise stated.

\begin{definition}
    Let $R$ be a ring and $I$ be an ideal of $R$. The \eax{quotient ring} $R/I$ is defined as the set of all cosets of $I$ in $R$; i.e.,
    \begin{align}
        R/I \defeq \{a + I : a \in R\}.
    \end{align}
    Addition and multiplication in $R/I$ are defined as follows: for all $a, b \in R$,
    \begin{align}
        (a + I) + (b + I) &\defeq (a + b) + I, \\
        (a + I) \cdot (b + I) &\defeq (a \cdot b) + I.
    \end{align}
\end{definition}

Of course, we must verify that these operations are well-defined. Note that $I$ is a normal subgroup of $(R, +)$ since $R$ is abelian under addition, so $R/I$ is an abelian group under addition. We verify that multiplication is well-defined as follows: let $a, a', b, b' \in R$ such that $a + I = a' + I$ and $b + I = b' + I$. Then, there exist $i_{1}, i_{2} \in I$ such that $a' = a + i_{1}$ and $b' = b + i_{2}$. Thus,
\begin{align}
    (a' \cdot b') + I &= ((a + i_{1}) \cdot (b + i_{2})) + I = (a \cdot b + a \cdot i_{2} + i_{1} \cdot b + i_{1} \cdot i_{2}) + I \notag \\
    &= (a \cdot b) + I,
\end{align}
since $a \cdot i_{2}, i_{1} \cdot b, i_{1} \cdot i_{2} \in I$. Therefore, multiplication is well-defined. Moreover, it is easy to verify that $R/I$ is a ring with unity under these operations, where the additive identity is $0 + I$ and the multiplicative identity is $1 + I$. One also has the \eax{quotient map} naturally defined as
\begin{align}
    q:R \to R/I, \quad q(a) = a + I \quad \text{ for all } a \in R.
\end{align}
It is easy to verify that $q$ is a ring epimorphism with kernel $I$. The most common example of a quotient ring is $\Z/n\Z$, which is isomorphic to the quotient ring $\Z/(n\Z)$.

\begin{example}
    In the ring $\Q[x]$, let $I = (x^{2}-2) = (x^{2}-2)\Q$. Then, the quotient $\Q[x]/(x^{2}-2)$ is indeed a quotient ring. It is also a field, and may be written as $\Q[\sqrt{2}]$. However, the quotient ring $\R[x]/(x^{2}-2)$ is not an integral domain since $(x- \sqrt{2} + I)(x + \sqrt{2} + I) = x^{2} - 2 + I = I$.
\end{example}

We are now fit to show the isomorphism theorems for rings.

\begin{theorem}[The \eax{first isomorphism theorem} for rings]
    Let $\varphi : R \to S$ be a ring homomorphism. Let $I = \ker \varphi$. Then there exists a unique ring monomorphism $\overline{\varphi} : R/I \to S$ such that $\varphi = \overline{\varphi} \circ q$, where $q : R \to R/I$ is the quotient map. Moreover, if $\varphi$ is surjective, then $\overline{\varphi}$ is a ring isomorphism.
\end{theorem}
\begin{proof}
    Define the map $\overline{\varphi} : R/I \to S$ as follows: for all $a + I \in R/I$, let
    \begin{align}
        \overline{\varphi}(a + I) = \varphi(a).
    \end{align}
    We must verify that this map is well-defined. Let $a, b \in R$ such that $a + I = b + I$. Then, there exists some $i \in I$ such that $b = a + i$. Thus,
    \begin{align}
        \varphi(b) = \varphi(a + i) = \varphi(a) + \varphi(i) = \varphi(a) + 0_{S} = \varphi(a),
    \end{align}
    so $\overline{\varphi}$ is well-defined. It is easy to verify that $\overline{\varphi}$ is a ring homomorphism. Also, for all $a \in R$,
    \begin{align}
        (\overline{\varphi} \circ q)(a) = \overline{\varphi}(a + I) = \varphi(a),
    \end{align}
    so $\varphi = \overline{\varphi} \circ q$.

    Now, suppose that $\overline{\varphi}(a + I) = 0_{S}$ for some $a + I \in R/I$. Then, $\varphi(a) = 0_{S}$, so $a \in I$. Thus, $a + I = I$, which is the additive identity in $R/I$. Therefore, $\overline{\varphi}$ is injective.

    Finally, if $\varphi$ is surjective, then for any $s \in S$, there exists some $a \in R$ such that $\varphi(a) = s$. Thus,
    \begin{align}
        \overline{\varphi}(a + I) = \varphi(a) = s,
    \end{align}
    so $\overline{\varphi}$ is surjective as well. Therefore, $\overline{\varphi}$ is a ring isomorphism.
\end{proof}

\begin{proposition}
    Let $R$ be a ring and $I$ be an ideal of $R$. Then there is a bijection between the set of all ideals of $R$ containing $I$ and the set of all ideals of the quotient ring $R/I$.
\end{proposition}
\begin{proof}
    We make use of the quotient map $q:R \to R/I$. Let $J$ be an ideal of $R$ such that $I \subseteq J$. The bijection is given by sending $J$ to $J/I \defeq q(J) = \{a + I : a \in J\}$, and sending $K$, an ideal of $R/I$, to $q^{-1}(K) = \{a \in R : q(a) \in K\}$. We first show that $q(J) = J/I$ is indeed an ideal of $R/I$, for $J$ an ideal of $R$ containing $I$. Let $x+I \in J/I$ and $r + I \in R/I$. Then, $(r+I)(x+I) = (r \cdot x) + I$. Since $x \in J$ and $J$ is an ideal of $R$, we have $r \cdot x \in J$, so $(r + I)(x + I) \in J/I$. Also note that for all $x + I, y + I \in J/I$, we have $(x + I) + (y + I) = (x + y) + I \in J/I$ since $x, y \in J$ and $J$ is an ideal of $R$. Thus, $J/I$ is an ideal of $R/I$.

    On the other hand, we show that $q^{-1}(K)$ is an ideal of $R$ for $K$ an ideal of $R/I$. Let $x,y \in q^{-1}(K)$. Then, $q(x), q(y) \in K$, so $q(x+y) = q(x) + q(y) \in K$ since $K$ is an ideal of $R/I$. Thus, $x + y \in q^{-1}(K)$. Also, for any $r \in R$ and $x \in q^{-1}(K)$, we have $q(r), q(x) \in R/I$ and $q(x) \in K$, so $q(r \cdot x) = q(r) \cdot q(x) \in K$ since $K$ is an ideal of $R/I$. Thus, $r \cdot x \in q^{-1}(K)$. Therefore, $q^{-1}(K)$ is an ideal of $R$. Also, if $x \in I$, then $q(x) = x + I = I$, which is the additive identity in $R/I$ and thus belongs to every ideal of $R/I$. Therefore, $I \subseteq q^{-1}(K)$.

    To show that the maps are inverses of each other is left as an exercise.
\end{proof}

\begin{theorem}[The \eax{second isomorphism theorem} for rings]
    Let $R$ be a ring, and let $S \subseteq R$ be a subring containing the unity. Let $I$ be an ideal of $R$. Then, $S + I = \{s + i : s \in S, i \in I\}$ is a subring of $R$ containing the unity, $S \cap I$ is an ideal of $S$, and there is a ring isomorphism
    \begin{align}
        (S + I)/I \cong S/(S \cap I).
    \end{align}
\end{theorem}
\begin{proof}
    Let $\alpha,\beta \in S+I$. Then $\alpha = s+x$ and $\beta = s'+y$ for some $s,s' \in S$ and $x,y \in I$. Thus, $\alpha + \beta = (s + s') + (x + y) \in S + I$ since $S$ is a subring and $I$ is an ideal. Also, $\alpha \cdot \beta = (s + x)(s' + y) = ss' + sy + xs' + xy \in S + I$ since $ss' \in S$, $sy, xs', xy \in I$. Therefore, $S + I$ is a subring of $R$ containing the unity.

    Note that the inclusion map $i:S \to R$ is a ring homomorphism. Thus, by the proposition above, $S \cap I = i^{-1}(I)$ is an ideal of $S$. Also, $I \subseteq S+I$ is an ideal of $S+I$. Now let $\varphi : S \to (S+I)/I$ be the map $\varphi = q \circ i$, where $q : S + I \to (S + I)/I$ is the quotient map. It is easy to verify that $\varphi$ is a ring homomorphism with kernel
    \begin{align}
        \ker \varphi = \{a \in S \mid q \circ i(a) = I\} = \{a \in S \mid a + I = I\} = S \cap I.
    \end{align}
    Moreover, $\varphi$ is surjective since for any $s + i + I \in (S + I)/I$ where $s \in S$ and $i \in I$, we have $\varphi(s) = s + I = s + i + I$. Thus, by the first isomorphism theorem, we have the desired isomorphism.
\end{proof}

\textit{January 30th.}

\begin{theorem}[The \eax{third isomorphism theorem} for rings]
    Let $R$ be a ring, and let $J \subseteq I$ be two ideals of $R$. Then, $I/J = \{a + J : a \in I\}$ is an ideal of the quotient ring $R/J$, and there is a ring isomorphism
    \begin{align}
        (R/J)/(I/J) \cong R/I.
    \end{align}
\end{theorem}
\begin{proof}
    Let $q_{R} : R \to R/J$ be the quotient map, and $q_{R/J} : R/J \to (R/J)/(I/J)$ be the quotient map. Thus, the composition $\varphi = q_{R/J} \circ q_{R} : R \to (R/J)/(I/J)$ is a surjective ring homomorphism. The kernel of $\varphi$ is given by
    \begin{align}
        \ker \varphi = \{x \in R \mid \varphi(x) = J + I/J\} = \{x \in R \mid q_{R}(x) \in I/J\} = \{x \in R \mid x + J \in I/J\} = I.
    \end{align}
    Thus, by the first isomorphism theorem, we have the desired isomorphism.
\end{proof}

We look at some applications of the isomorphism theorems.

\begin{example}
    Let $I = (5) \subseteq \Z[x]$. We claim that $\Z[x]/5\Z[x] \cong (\Z/5\Z)[x]$. To see this, we make use of the first isomorphism theorem. Let $\varphi : \Z[x] \to (\Z/5\Z)[x]$ be the map defined by
    \begin{align}
        \varphi\left( a_{0} + a_{1}x + a_{2}x^{2} + \cdots + a_{n}x^{n} \right) = \bar{a}_{0} + \bar{a}_{1}x + \bar{a}_{2}x^{2} + \cdots + \bar{a}_{n}x^{n},
    \end{align}
    where $\bar{a}_{i}$ is the image of $a_{i}$ in $\Z/5\Z$ for all $0 \leq i \leq n$. It is easy to verify that $\varphi$ is a surjective ring homomorphism with kernel $5\Z[x]$. Thus, by the first isomorphism theorem, we have the desired isomorphism.
\end{example}

\begin{example}
    Let $(x) \subseteq \Z[x]$. We claim that $\Z[x]/(x) \cong \Z$. To see this, we make use of the second isomorphism theorem. Let $S = \Z \subseteq \Z[x]$. Then, $S + (x) = \Z[x]$ since for any $f(x) \in \Z[x]$, we have $f(x) = f(0) + (f(x) - f(0)) \in S + (x)$. Also, $S \cap (x) = \{0\}$ since the only constant polynomial in $(x)$ is the zero polynomial. Thus, by the second isomorphism theorem, we have
    \begin{align}
        \Z[x]/(x) \cong S/(S \cap (x)) = S/\{0\} \cong \Z.
    \end{align}
\end{example}

\begin{example}
    Again, let $I = (x^{2}-4,2) \subseteq \Z[x]$. We claim the isomorphism
    \begin{align}
        \Z[x]/(x^{2}-4,2) \cong \Z/2\Z[x]/(x^{2}).
    \end{align}
    To see this, we make use of the third isomorphism theorem. Let $J = (2) \subseteq I$. Then, by the third isomorphism theorem, we have
    \begin{align}
        \Z[x]/I \cong (\Z[x]/J)/(I/J) \cong (\Z/2\Z)[x]/(x^{2}-4 + J) = (\Z/2\Z)[x]/(x^{2}),
    \end{align}
    since $x^{2}-4 + J = x^{2} + J$ in $(\Z/2\Z)[x]$.
\end{example}


\section{Prime and Maximal Ideals}

\begin{definition}
    Let $R$ be a ring. An ideal $P \subseteq R$ is called a \eax{prime ideal} if $P \neq R$ and for all $a, b \in R$ such that $a \cdot b \in P$, we have either $a \in P$ or $b \in P$.
\end{definition}

Of course, the most common example of a prime ideal is $(0_{R})$ in an integral domain $R$. Another example is $(p) = p\Z$ in $\Z$ for any prime $p$. Note that if $R$ is a field, then the only prime ideal of $R$ is $(0_{R})$.

\begin{theorem}
    Let $I$ be an ideal of a ring $R$. Then, $I$ is a prime ideal if and only if the quotient ring $R/I$ is an integral domain.
\end{theorem}
\begin{proof}
    Suppose that $R/I$ is an integral domain. Let $a, b \in R$ such that $a \cdot b \in I$. Then,
    \begin{align}
        (a + I)(b + I) = (a \cdot b) + I = I,
    \end{align}
    which is the zero element in $R/I$. Since $R/I$ is an integral domain, either $a + I = I$ or $b + I = I$, which implies that either $a \in I$ or $b \in I$. Thus, $I$ is a prime ideal. If we now suppose that $I$ is a prime ideal, let $a + I, b + I \in R/I$ such that $(a+I)(b+I) = (a \cdot b) + I = I$. This implies that $a \cdot b \in I$, so either $a \in I$ or $b \in I$. Thus, either $a + I = I$ or $b + I = I$, so $R/I$ is an integral domain.
\end{proof}

One can also show that there is the natural bijection between ideals of $R/I$ and ideals of $R$ containing $I$ restricts to a bijection between prime ideals of $R/I$ and prime ideals of $R$ containing $I$.

\begin{example}
    We can use this theorem to show $(x^{2}+1)$ is a prime ideal of $\Z[x]$. Indeed, look at the ring homomorphism $\varphi : \Z[x] \to \Z[i]$ defined by $\varphi(f) = f(i)$ for all $f \in \Z[x]$. For the kernel, let $f \in \ker \varphi$. By the division algorithm, there exist unique $q, r \in \Z[x]$ such that
    \begin{align}
        f(x) = (x^{2}+1)q(x) + r(x),
    \end{align}
    where either $r(x) = 0$ or $\deg r < 2$. Plugging in $x = i$ gives $f(i) = 0 = r(i)$. The only way an at most linear polynomial $r(x)$ can be 0 at $x = i$ is if $r$ is the zero polynomial. Hence, $\ker \varphi = (x^{2}+1)$. Since $\Z[i]$ is an integral domain, by the first isomorphism theorem, we have
    \begin{align}
        \Z[x]/(x^{2}+1) \cong \Z[i]
    \end{align}
    showing that $(x^{2}+1)$ is a prime ideal of $\Z[x]$.
\end{example}

Another notion is the maximal ideal.

\begin{definition}
    Let $R$ be a ring. An ideal $M \subseteq R$ is called a \eax{maximal ideal} if $M \neq R$ and there are no ideals $I$ of $R$ such that $M \subsetneq I \subsetneq R$.
\end{definition}

That is, if $J$ is an ideal such that $M \subseteq J$, then either $J = M$ or $J = R$. For example, in $\Z$, the ideals $(p) = p\Z$ for prime $p$ are maximal ideals. Note that if $R$ is a field, then the only maximal ideal of $R$ is $(0_{R})$. In fact, it may be shown that $R$ is a field if and only if $(0)$ and $R$ are the only ideals.\\

\textit{February 2nd.}
\begin{theorem}
    Let $I$ be an ideal of a ring $R$. Then, $I$ is a maximal ideal if and only if the quotient ring $R/I$ is a field.
\end{theorem}
\begin{proof}
    We work with a set of equivalences. $I \subseteq R$ is a maximal ideal \(\iff\) $\{J \mid I \subseteq J \subseteq R\} = \{I, R\}$ \(\iff\) $\{K \mid K \text{ is an ideal of } R/I\} = \{I/I, R/I\}$ \(\iff\) $R/I$ is a field.
\end{proof}

A neat corollary of this theorem is that every maximal ideal is a prime ideal. Indeed, if $M$ is a maximal ideal of $R$, then $R/M$ is a field, and thus an integral domain. Therefore, by the previous theorem, $M$ is a prime ideal. Another theorem guarantees the existence of maximal ideals.

\begin{theorem}
    Every non-zero ring has at least a maximal ideal.
\end{theorem}
This theorem, though true in many common cases, requires Zorn's lemma for a general proof.
\begin{proof}
    Let $(\Omega, \subseteq)$ be the set of all proper ideals of a ring $R$, ordered by inclusion. Note that $R \supsetneq (0) \in \Omega$, so $\Omega \neq \emptyset$. Zorn's lemma states that if every chain in $\Omega$ has an upper bound in $\Omega$, then $\Omega$ has a maximal element. Let $\cC$ be a chain in $\Omega$. We claim that $I_{\cC} = \bigcup_{I \in \cC} I$ is an upper bound of $\cC$ in $\Omega$. Certainly, for any $I \in \cC$, we have $I \subseteq J$. Also, we must verify that $I_{\cC}$ is a proper ideal of $R$.

    Let $x,y \in I_{\cC}$, and $r \in R$. Then there exist two ideals $I,J \in \cC$ such that $x \in I$ and $y \in J$. Since $\cC$ is a chain, without loss of generality, suppose that $I \subseteq J$. Thus, $x,y \in J$, so $x + y \in R$ and $r \cdot x \in R$ since $J$ is an ideal of $R$. Therefore, $I_{\cC}$ is an ideal of $R$. Also, if $I_{\cC} = R$, then $1_{R} \in I_{\cC}$, so there exists some ideal $I \in \cC$ such that $1_{R} \in I$. This implies that $I = R$, which contradicts the fact that $I$ is a proper ideal. Thus, $I_{\cC}$ is a proper ideal of $R$, so $I_{\cC} \in \Omega$. Therefore, by Zorn's lemma, $\Omega$ has a maximal element, which is a maximal ideal of $R$.
\end{proof}

\begin{example}
    In $\C[x]$, the ideal $(x)$ is maximal since $\C[x]/(x) \cong \C$, which is a field. However, $(x^{2})$ is not maximal since $(x^{2}) \subsetneq (x)$. Another reasoning is that $\C[x]/(x^{2})$ is not a field since $(x + (x^{2})) \cdot (x + (x^{2})) = x^{2} + (x^{2}) = 0 + (x^{2})$; it is not an integral domain either, so $(x^{2})$ is not even prime. In fact, the maximal ideals of $\C[x]$ are precisely of the form $(x - a)$ for some $a \in \C$.
\end{example}

\subsection{Jacobson Radical and Nilradical}
\textit{February 5th.}


\begin{definition}
    The \eax{Jacobson radical} of a non-zero ring $R$ is defined as
    \begin{align}
        \Jac(R) = \bigcap_{\substack{\mf{m} \subseteq R \\ \mf{m} \text{ is a maximal ideal}}} \mf{m}.
    \end{align}
    It is the interseciton of ideals, so it is also an ideal. The \eax{nilradical} of $R$ is defined as
    \begin{align}
        \nil(R) = \{x \in R \mid x^{n} = 0 \text{ for some } n \in \N\}.
    \end{align}
\end{definition}


The following are some properties of the Jacobson radical and nilradical.
\begin{proposition}
    Let $R$ be a ring. Then,
    \begin{enumerate}
        \item $\nil(R)$ is an ideal of $R$.
        \item $\nil(R) \subseteq \Jac(R)$.
        \item $x \in \Jac(R)$ if and only if $1+ax$ is a unit for all $a \in R$.
        \item $\nil(R) = \bigcap_{\substack{P \subseteq R \\ P \text{ is a prime ideal}}} P$.
    \end{enumerate}
\end{proposition}

\begin{proof}
    \begin{enumerate}
        \item Let $x,y \in \nil(R)$ and $r \in R$. Then, there exist $m,n \in \N$ such that $x^{m} = 0$ and $y^{n} = 0$. Thus, $(x + y)^{m+n} = 0$ by the binomial theorem, so $x + y \in \nil(R)$. Also, $(r \cdot x)^{m} = r^{m} \cdot x^{m} = 0$, so $r \cdot x \in \nil(R)$. Therefore, $\nil(R)$ is an ideal of $R$.
        \item Let $x \in \nil(R)$. Then, there exists some $n \in \N$ such that $x^{n} = 0$. Thus $x^{n}$ belongs to every ideal of $R$, so in particular, $x^{n} \in P$ for every prime ideal and $x^{n} \in \mf{m}$ for every maximal ideal. However, $x^{n} \in P$ implies that $x \in P$ since $P$ is prime, and $x^{n} \in \mf{m}$ implies that $x \in \mf{m}$ since $\mf{m}$ is prime as well. Therefore, $x$ belongs to every prime ideal and every maximal ideal, so $x \in \Jac(R)$. Hence, $\nil(R) \subseteq \Jac(R)$.
        \item For the forward implication, let $x \in \Jac(R)$ and $a \in R$. Note that $ax$ cannot be a unit since it is contained in the (maximal) ideals $\mf{m}$. Suppose $1+ax$ is \textit{not} a unit. Then $I = (1+ax)$ is a proper ideal of $R$. If $q : R \to R/I$ is the quotient map, and $\bar{\mf{m}}$ is a maximal ideal of $R/I$, then we claim that $\mf{m} = q^{-1}(\bar{\mf{m}})$ is a maximal ideal of $R$ containing $I$. Certainly, $\mf{m}$ is an ideal of $R$ containing $I$ by the proposition above. Also, if there exists some ideal $J$ of $R$ such that $\mf{m} \subsetneq J \subsetneq R$, then $I \subseteq J \subsetneq R$, so $q(J)$ is an ideal of $R/I$ such that $\bar{\mf{m}} \subsetneq q(J) \subsetneq R/I$, contradicting the maximality of $\bar{\mf{m}}$. Thus, $\mf{m}$ is a maximal ideal of $R$. However, since $x \in \Jac(R)$, we have $x \in \mf{m}$, so $ax \in I \subseteq \mf{m}$. This implies that $1 = (1 + ax) - ax \in \mf{m}$, a contradiction. Therefore, $1 + ax$ is a unit for all $a \in R$.
        
        For the converse, suppose $1+ax$ is a unit for all $a \in R$. Let $\mf{m}$ be any maximal ideal of $R$. If $x \notin \mf{m}$, then the ideal $(\mf{m}, x)$ properly contains $\mf{m}$, so $(\mf{m}, x) = R$. Thus, there exist $m \in \mf{m}$ and $r \in R$ such that $1 = m + r \cdot x$. This implies that $1 - r \cdot x = m \in \mf{m}$, contradicting the fact that $1 - r \cdot x$ is a unit. Therefore, $x \in \mf{m}$. Since $\mf{m}$ was an arbitrary maximal ideal, we have $x \in \Jac(R)$.

        \item It is clear that $\nil(R) \subseteq \bigcap_{P \text{ prime}} P$ since maximal ideals are prime ideals. For the converse inclusion, let $x \in \bigcap_{P \text{ prime}} P$. We claim that $x$ is nilpotent. Suppose not. Then, the set $S = \{x^{n} : n \in \N\}$ does not contain $0$. Define the set $\Omega = \{I \subseteq R \mid I \text{ is an ideal and } I \cap S = \emptyset\}$. Inclusion $\subseteq$ is a partial order on $\Omega$. Let $\cC$ be a chain in $\Omega$ and let $I = \bigcup_{J \in \cC} J$. We claim that $I$ is an upper bound of $\cC$ in $\Omega$. It is an ideal since if $z,y \in I$, then $z \in J_{1}$ and $y \in J_{2}$ with $J_{1} \subseteq J_{2}$. Thus, $z,y \in J_{1} \cup J_{2}$, so $z + y \in J_{1} \cup J_{2} \subseteq I$. Also, for any $r \in R$ and $z \in I$, we have $z \in J$ for some $J \in \cC$, so $r \cdot z \in J \subseteq I$. Therefore, $I$ is an ideal. Moreover, if $J \cap S = \emptyset$ for all $J \in \cC$ tells us $I \cap S = \emptyset$. Hence, $I$ is a valid upper bound, and Zorn's lemma guarantess the existence of a maximal $P$ in $\Omega$. We further claim that $P$ is a prime ideal of $R$. Indeed, $P \subsetneq R$ since $P \cap S = \emptyset$. Let $uv \in P$ for some $u,v \in R$. If both $u,v \notin P$, then the ideals $(P,u)$ and $(P,v)$ properly contain $P$; that is, $au + y = x^{n}$ and $bv + z = x^{m}$ for some $a,b \in R$, $y,z \in P$, and $n,m \in \N$. Thus, $(au+y)(bv+z) = x^{n+m} \in S$, but $(au+y)(bv+z) = abuv + az + by + yz \in P$ since $uv \in P$ and $y,z \in P$. This contradicts the fact that $P \cap S = \emptyset$. Therefore, either $u \in P$ or $v \in P$, so $P$ is a prime ideal. However, by construction, $x$ is in every prime ideal of $R$, so $x \in P$, contradicting the fact that $P \cap S = \emptyset$. Hence, $x$ is nilpotent, so $x \in \nil(R)$.
    \end{enumerate}
\end{proof}

\noindent \textit{February 6th.}

Let us look at some examples.

\begin{example}
    In $\Z$, $\nil(\Z) = \{0\}$ since the only nilpotent element is $0$. Also, $\Jac(\Z) = \{0\}$ since the intersection of all maximal ideals $(p)$ for prime $p$ is $\{0\}$. In the ring $\Z/6\Z$, the nilradical is $\{0\}$ even though it is not an integral domain. The Jacobson radical is $\{0\}$ as well since the maximal ideals are $(2)$ and $(3)$, whose intersection is $\{0\}$.
\end{example}

\begin{example}
    For a non-trivial example, let us look at the ring $R = \Z/4\Z$. The nilradical of $R$ is $\{0, 2\}$ since $2^{2} = 0$ in $R$. The Jacobson radical of $R$ is also $\{0, 2\}$ since the only maximal ideal of $R$ is $(2)$. If we take $R = \Q[x,y]/(x^{2},y)$, then $\nil(R) = (x)$ since $x$ is nilpotent. The Jacobson radical of $R$ is also $(x)$.
\end{example}


\begin{example}
    For an example where the Jacobson radical is strictly larger than the nilradical, consider the power series ring $R = \Q[[x]]$. If $f \in \Q[[x]]$, then $f$ is a unit if and only if the constant term of $f$ is non-zero; if $f = a_{0}(1-g)$ where $a_{0} \in \Q$ and $g \in (x)$, then $f^{-1} = a_{0}^{-1}(1-g)^{-1}$, where $(1-g)^{-1}$ is given by the geometric series expansion $(1-g)^{-1} = 1 + g + g^{2} + \cdots$. Thus, $\Jac(R) = (x)$ since $1 + ax$ is a unit for all $a \in R$. However, $\nil(R) = \{0\}$ since there are no non-zero nilpotent elements in $R$.
\end{example}


\section{Product of Rings}

Let $R_{1},R_{2},\ldots,R_{n}$ be rings. We can define the \eax{product ring} $R_{1} \times R_{2} \times \cdots \times R_{n}$ as the set of all $n$-tuples $(r_{1}, r_{2}, \ldots, r_{n})$ where $r_{i} \in R_{i}$ for all $1 \leq i \leq n$, with addition and multiplication defined componentwise. It is easy to verify that $R_{1} \times R_{2} \times \cdots \times R_{n}$ is also a ring, and the unity is given by $(1_{R_{1}}, 1_{R_{2}}, \ldots, 1_{R_{n}})$. The \eax{projection map} $p_{i} : R_{1} \times R_{2} \times \cdots \times R_{n} \to R_{i}$ is defined by $p_{i}(r_{1}, r_{2}, \ldots, r_{n}) = r_{i}$ for all $1 \leq i \leq n$. It is, again, easy to verify that $p_{i}$ is a surjective ring homomorphism with kernel $R_{1} \times \cdots \times R_{i-1} \times \{0_{R_{i}}\} \times R_{i+1} \times \cdots \times R_{n}$. Conversely, the \eax{inclusion map} $e_{i} : R_{i} \to R_{1} \times R_{2} \times \cdots \times R_{n}$ is defined by $e_{i}(r) = (0_{R_{1}}, \ldots, 0_{R_{i-1}}, r, 0_{R_{i+1}}, \ldots, 0_{R_{n}})$ for all $r \in R_{i}$. It is easy to verify that $e_{i}$ is an injective ring homomorphism with image $\{0_{R_{1}}\} \times \cdots \times \{0_{R_{i-1}}\} \times R_{i} \times \{0_{R_{i+1}}\} \times \cdots \times \{0_{R_{n}}\}$.

Regarding ideals, we have the following proposition.

\begin{proposition}
    Let $R_{1}, R_{2}, \ldots, R_{n}$ be rings. Then, every ideal of the product ring $R_{1} \times R_{2} \times \cdots \times R_{n}$ is of the form $I_{1} \times I_{2} \times \cdots \times I_{n}$ where $I_{i}$ is an ideal of $R_{i}$ for all $1 \leq i \leq n$.
\end{proposition}
\begin{proof}
    Let $I$ be an ideal of $R_{1} \times R_{2} \times \cdots \times R_{n}$. For each $1 \leq i \leq n$, let $I_{i} = p_{i}(I)$. We claim that each $I_{i}$ is an ideal of $R_{i}$ and $I = I_{1} \times I_{2} \times \cdots \times I_{n}$. Let $x_{i},y_{i} \in I_{i}$ and $r_{i} \in R_{i}$. Then, there exist $x = (x_{1}, \ldots, x_{n}), y = (y_{1}, \ldots, y_{n}) \in I$ such that $p_{i}(x) = x_{i}$ and $p_{i}(y) = y_{i}$. Thus, $x + y \in I$ since $I$ is an ideal of the product ring, so $x_{i} + y_{i} = p_{i}(x+y) \in I_{i}$. Also, $r_{i} \cdot x_{i} = p_{i}(e_{i}(r_{i}) \cdot x) \in I_{i}$ since $e_{i}(r_{i}) \cdot x \in I$. Therefore, $I_{i}$ is an ideal of $R_{i}$.

    Let use first show that $I \supseteq I_{1} \times I_{2} \times \cdots \times I_{n}$. Let $(a_{1}, a_{2}, \ldots, a_{n}) \in I_{1} \times I_{2} \times \cdots \times I_{n}$. Let $x^{(i)} \in I$ such that $p_{i}(x^{(i)}) = a_{i}$ for all $1 \leq i \leq n$. Then $e_{i}(1_{R_{i}}) \cdot x^{(i)} \in I$ since $I$ is an ideal of the product ring, and so $\sum_{i=1}^{n} e_{i}(1_{R_{i}}) \cdot x^{(i)} \in I$ as well. However, $\sum_{i=1}^{n} e_{i}(1_{R_{i}}) \cdot x^{(i)} = (a_{1}, a_{2}, \ldots, a_{n})$, so $(a_{1}, a_{2}, \ldots, a_{n}) \in I$. Therefore, $I \supseteq I_{1} \times I_{2} \times \cdots \times I_{n}$. For the converse inclusion $I \subseteq I_{1} \times I_{2} \times \cdots \times I_{n}$, let $(a_{1},a_{2}, \ldots, a_{n}) \in I$. Then, $a_{i} = p_{i}(a_{1}, a_{2}, \ldots, a_{n}) \in I_{i}$ for all $1 \leq i \leq n$. Thus, $(a_{1}, a_{2}, \ldots, a_{n}) \in I_{1} \times I_{2} \times \cdots \times I_{n}$. Therefore, $I = I_{1} \times I_{2} \times \cdots \times I_{n}$.
\end{proof}

We now wish to study the prime ideals of the product ring. 

\begin{proposition}
    Let $R_{1}, R_{2}, \ldots, R_{n}$ be rings. Then $I \subseteq R_{1} \times R_{2} \times \cdots \times R_{n}$ is a prime ideal if and only if $p_{i}(I) = R_{i}$ for all $1 \leq i \leq n$ except for one index $1 \leq k \leq n$ such that $p_{k}(I)$ is a prime ideal of $R_{k}$.
\end{proposition}

\begin{proof}
    For the forward direction, let $I = I_{1} \times \cdots I_{n}$ (by the previous proposition), where $I_{i} = p_{i}(I)$ is an ideal of $R_{i}$. Note that $I$ is prime if and only if $(R_{1} \times \cdots \times R_{n})/I \cong R_{1}/I_{1} \times \cdots R_{n}/I_{n}$ is an integral domain. This implies that $R_{i}/I_{i}$ is an integral domain for one index $1 \leq k \leq n$ and $R_{i}/I_{i} \cong \{0\}$ for all $i \neq k$. It it were otherwise, we could find two indices $i \neq j$ such that $R_{i}/I_{i}$ and $R_{j}/I_{j}$ are both non-trivial, and $e_{i}(1_{R_{i}}) + I$ and $e_{j}(1_{R_{j}}) + I$ are two non-zero elements in $R_{1}/I_{1} \times \cdots R_{n}/I_{n}$ whose product is zero, contradicting the fact that $R_{1}/I_{1} \times \cdots R_{n}/I_{n}$ is an integral domain. Thus, $p_{i}(I) = I_{i} = R_{i}$ for all $i \neq k$, and $p_{k}(I) = I_{k}$ is a prime ideal of $R_{k}$.

    Conversely, suppose that $p_{i}(I) = R_{i}$ for all $1 \leq i \leq n$ except for one index $1 \leq k \leq n$ such that $p_{k}(I)$ is a prime ideal of $R_{k}$. Then, $(R_{1} \times \cdots \times R_{n})/I \cong R_{1}/R_{1} \times \cdots R_{k}/p_{k}(I) \times \cdots R_{n}/R_{n} \cong R_{k}/p_{k}(I)$ is an integral domain since $p_{k}(I)$ is a prime ideal of $R_{k}$. Thus, $I$ is a prime ideal of the product ring.
\end{proof}

\subsection{Idempotents}

\begin{definition}
    Let $R$ be a ring. An element $x \in R$ is called an \eax{idempotent} if $x^{2} = x$.
\end{definition}

\begin{example}
    Trivially, every ring has two idempotents, namely $0_{R}$ and $1_{R}$. If $R_{1}$ and $R_{2}$ are rings, then $(1,1),(0,0),(1,0),(0,1)$ are idempotents of the product ring $R_{1} \times R_{2}$.
\end{example}

\begin{proposition}
    Let $R$ be a ring and $e \in R$ be an idempotent. Then $1-e$ is also an idempotent, and $R \cong eR \times (1-e)R$.
\end{proposition}

\begin{proof}
    Note that $(1-e)e = e-e^{2} = 0$, and $(1-e)^{2} = (1-e)(1-e) = 1-e$. To show the isomorphism, let $\varphi : R \to eR \times (1-e)R$ be defined by $\varphi(r) = (er, (1-e)r)$ for all $r \in R$. It is easy to verify that $\varphi$ is a ring homomorphism. If we define $\psi : eR \times (1-e)R \to R$ by $\psi(x,y) = x + y$ for all $x \in eR$ and $y \in (1-e)R$, then $\psi$ is also a ring homomorphism. Moreover, $\psi \circ \varphi(r) = \psi(er, (1-e)r) = er + (1-e)r = r$ for all $r \in R$, and $\varphi \circ \psi(x,y) = \varphi(x+y) = (e(x+y), (1-e)(x+y)) = (x,y)$ for all $x \in eR$ and $y \in (1-e)R$. Thus, $\varphi$ is an isomorphism, so $R \cong eR \times (1-e)R$.
\end{proof}

\noindent \textit{February 9th.}

An interesting consequence is the chinese remainder theorem, which is abstracted from the original number-theoretic version. For the below theorem, we define the product of two ideals $I$ and $J$ of a ring $R$ as
\begin{align}
    IJ = \{a_{1}b_{1} + \cdots + a_{n}b_{n} \mid a_{i} \in I, b_{i} \in J, n \in \N\}.
\end{align}
One may verify that $IJ$ is an ideal of $R$. Moreover, $IJ \subseteq I \cap J$ since $a_{i}b_{i} \in I$ and $a_{i}b_{i} \in J$ for all $1 \leq i \leq n$. However, it is not necessarily the case that $IJ = I \cap J$. For example, if $R = \Z$ and $I = (2)$ and $J = (4)$, then $IJ = (8)$ but $I \cap J = (4)$. 

\begin{theorem}[The \eax{chinese remainder theorem}]
    Let $R$ be a ring, and $I_{1}, I_{2}, \ldots, I_{k}$ be ideals of $R$ which are pairwise co-maximal; that is, for all $1 \leq i \neq j \leq k$, we have $I_{i} + I_{j} = R$. Then $I_{1} \cap I_{2} \cap \cdots \cap I_{k} = I_{1} I_{2} \cdots I_{k}$. Moreover, the natural homomorphism $\varphi : R \to R/I_{1} \times R/I_{2} \times \cdots \times R/I_{k}$ defined by $\varphi(r) = (r + I_{1}, r + I_{2}, \ldots, r + I_{k})$ is surjective with kernel $I_{1} I_{2} \cdots I_{k}$, so
    \begin{align}
        R/(I_{1} I_{2} \cdots I_{k}) \cong R/I_{1} \times R/I_{2} \times \cdots \times R/I_{k}.
    \end{align}
\end{theorem}

We first determine how this abstract version of the chinese remainder theorem implies the original number-theoretic version. Let $n_{1}, n_{2}, \ldots, n_{k}$ be pairwise coprime positive integers, and let $m = n_{1} n_{2} \cdots n_{k}$. Then, the ideals $(n_{1}), (n_{2}), \ldots, (n_{k})$ of $\Z$ are pairwise co-maximal since $(n_{i}) + (n_{j}) = (1)$ for all $1 \leq i \neq j \leq k$. Thus, by the chinese remainder theorem, we have
\begin{align}
    \Z \to \Z/n_{1}\Z \times \Z/n_{2}\Z \times \cdots \times \Z/n_{k}\Z, \quad x \mapsto (x + n_{1}\Z, x + n_{2}\Z, \ldots, x + n_{k}\Z)
\end{align}
is surjective with kernel $(m)$. That is, given any $(a_{1} + n_{1}\Z, a_{2} + n_{2}\Z, \ldots, a_{k} + n_{k}\Z) \in \Z/n_{1}\Z \times \Z/n_{2}\Z \times \cdots \times \Z/n_{k}\Z$, there exists some integer $x$ such that $x \equiv a_{i} \pmod{n_{i}}$ for all $1 \leq i \leq k$, and any two such integers are congruent modulo $m$. Moreover, we have the isomorphism
\begin{align}
    \Z/m\Z \cong \Z/n_{1}\Z \times \Z/n_{2}\Z \times \cdots \times \Z/n_{k}\Z.
\end{align}

\begin{proof}
    We first show for $k = 2$. Let $I_{1}$ and $I_{2}$ be two ideals of $R$ such that $I_{1} + I_{2} = R$. Then, there exist $a \in I_{1}$ and $b \in I_{2}$ such that $a + b = 1$. We already know that $I_{1} I_{2} \subseteq I_{1} \cap I_{2}$, so we only need to show the converse inclusion. Let $x \in I_{1} \cap I_{2}$. Then, $x = x(a + b) = xa + xb \in I_{1} I_{2}$ since $xa \in I_{1} I_{2}$ and $xb \in I_{1} I_{2}$. Therefore, $I_{1} \cap I_{2} = I_{1} I_{2}$. We now wish to show the natural map $R \to R/I_{1} \times R/I_{2}$ defined by $r \mapsto (r + I_{1}, r + I_{2})$ is surjective with kernel $I_{1} I_{2}$. It is easy to verify that the map is a ring homomorphism. For surjectivity, since any element of $R/I_{1} \times R/I_{2}$ is of the form $(r_{1} + I_{1}, r_{2} + I_{2})$ for some $r_{1}, r_{2} \in R$, we only show that $(1+I_{1}, 0 + I_{2})$ and $(0 + I_{1}, 1 + I_{2})$ are in the image of the map. Since $a \in I_{1}$, we have $a + I_{1} = 0 + I_{1}$, so $\varphi(b) = (b + I_{1}, b + I_{2}) = (1 + I_{1}, 0 + I_{2})$. Similarly, since $b \in I_{2}$, we have $b + I_{2} = 0 + I_{2}$, so $\varphi(a) = (a + I_{1}, a + I_{2}) = (0 + I_{1}, 1 + I_{2})$. Therefore, the map is surjective. For the kernel, if $r \in R$ is such that $\varphi(r) = (r + I_{1}, r + I_{2}) = (0 + I_{1}, 0 + I_{2})$, then $r \in I_{1} \cap I_{2} = I_{1}I_{2}$. Conversely, if $r \in I_{1}I_{2}$, then $r \in I_{1} \cap I_{2}$, so $\varphi(r) = (0 + I_{1}, 0 + I_{2})$. Thus, the kernel of the map is $I_{1}I_{2}$. For arbitrary $k$, we claim that $I_{1}$ and $I_{2}\cdots I_{k}$ are co-maximal ideals. Since $I_{1}$ and $I_{j}$ are co-maximal for all $2 \leq j \leq k$, there exist $x_{j} \in I_{1}$ and $y_{j} \in I_{j}$ such that $x_{j} + y_{j} = 1$ for all $2 \leq j \leq k$. Then 
    \begin{align}
        (x_{2}+y_{2})(x_{3}+y_{3}) \cdots (x_{k}+y_{k}) = 1 \implies \alpha + y_{2}y_{3} \cdots y_{k} = 1
    \end{align}
    where $\alpha \in I_{1}$ since $x_{j} \in I_{1}$ for all $2 \leq j \leq k$, and $y_{2} \cdots y_{k} \in I_{2} \cdots I_{k}$. Thus, $I_{1}$ and $I_{2} \cdots I_{k}$ are co-maximal. Now induction can be applied on $k$ to show $I_{1} \cap I_{2} \cap \cdots \cap I_{k} = I_{1} \cap I_{2} \cdots I_{k} = I_{1} I_{2} \cdots I_{k}$. To show surjectivity of the natural map $R \to R/I_{1} \times R/I_{2} \times \cdots \times R/I_{k}$, we show $e_{j} \defeq (0 + I_{1}, \ldots, 0 + I_{j-1}, 1 + I_{j}, 0 + I_{j+1}, \ldots, 0 + I_{k})$ is in the image of the map for all $1 \leq j \leq k$. Since $I_{j}$ and $I_{1} \cdots I_{j-1} I_{j+1} \cdots I_{k}$ are co-maximal, there exist $x \in I_{j}$ and $y \in I_{1} \cdots I_{j-1} I_{j+1} \cdots I_{k}$ such that $x + y = 1$. Then $\varphi(y) = (y + I_{1}, y + I_{2}, \ldots, y + I_{k}) = e_{j}$ since $y \in I_{i}$ for all $i \neq j$ and $y + I_{j} = 1 + I_{j}$. Therefore, the natural map is surjective with kernel $I_{1}I_{2} \cdots I_{k}$, so we have the desired isomorphism $R/(I_{1} I_{2} \cdots I_{k}) \cong R/I_{1} \times R/I_{2} \times \cdots \times R/I_{k}$.
\end{proof}