\chapter{ELECTROSTATICS}

\section{Introduction}

\textit{January 5th.}

The bulk of classical mechanics was developed by Newton in the 17th century. It was mostly concerned with the motion and trajectories of macroscopic objects under the influence of forces. This theory, however, breaks down at very high speeds when $v/c$ approaches unity. This necessitated the special theory of relativity by Einstein in 1905. Electrodynamics is a part of classical mechanics, but sits at the crossroads of both special relativity and quantum mechanics; it does sit in classical mechanics yet violates many Newtonian principles. The idea of Einstein's special relativity paper was born with these conflicts of electrodynamics with Newtonian mechanics.

Another area where classical mechanics fails is at very small length scales. The laws of classical mechanics are unable to explain phenomena at atomic and subatomic scales, leading to the development of quantum mechanics in the early 20th century. Quantum mechanics and special relaitivity were later unified into quantum field theory. Under this, electrodynamics was reformulated as quantum electrodynamics.

\subsection{Forces}

Every force in nature can be classified into four fundamental interactions:
\begin{itemize}
    \item Gravitational force
    \item Electromagnetic force
    \item Strong force
    \item Weak force
\end{itemize}
The \eax{strong force} is reponsible for holding the nucleus of an atom together, while the \eax{weak force} is responsible for nuclear decay. They are both short-range forces, acting only at subatomic distances. The \eax{gravitational force} is the weakest of the four fundamental forces, but it has an infinite range and acts attractively on all masses. The \eax{electromagnetic force} is much stronger than gravity and also has an infinite range, but it can be both attractive and repulsive, acting on charged particles.

If we set two electrons one metre apart, the gravitational force between them is approximately $10^{-42}$ times weaker than the electromagnetic force. This stark contrast highlights the relative weakness of gravity compared to electromagnetism at the scale of elementary particles. The electromagnetic force has an important aspect in that it unifies electricity and magnetism, which were once thought to be separate phenomena.


\subsection{Electrostatics}

The field of \eax{electrostatics} deals with the study of electric charges at rest. The goal is, given a set of source charges with their positions and magnitudes, to determine the force on a test charge with a given position and magnitude. Moreover, we want to determine the trajectory of the test charge. The principle of superposition plays a crucial role in electrostatics, allowing us to calculate the net electric field or force by summing the contributions from individual charges. In general, both source and test charges can be in motion; the net force is not so simple to calculate in that case, since the individual forces also depends on the velocities of the charges and even their accelerations.

It is not sufficient to just know positions and velocities at present time; we also need to know them at an earlier time due to the fact that electromagnetic interactions ``news'' travel at a finite velocity. We simplify this situation greatly. First, we assume that all source charges are at rest at fixed positions; they are stationary. Second, we assume that any electromagnetic effects propagate instantaneously.

With these assumptions, we can focus on the electrostatic forces between charges. The fundamental law governing these interactions is \eax{Coulomb's law}. Let there be a point charge $q$ located at position $\bv{r}'$ and a test charge $Q$ located at position $\bv{r}$. If $\bcurs{r} = \bv{r} - \bv{r}'$ is the displacement vector from the source charge to the test charge, then the force $\bv{F}$ on the test charge due to the source charge is given by

\begin{align}
    \bv{F} = \frac{1}{4 \pi \epsilon_{0}} \frac{q Q}{\curs{r}^{2}} \hat{\bcurs{r}},
\end{align}

where $\epsilon_{0}$ is the \eax{permittivity of free space}, a fundamental constant with a value of approximately $8.854 \times 10^{-12} \; \mathrm{C^{2}/Nm^{2}}$ in SI units. The unit vector $\hat{\bcurs{r}}$ points from the source charge to the test charge, indicating the direction of the force. The force is attractive if the charges have opposite signs and repulsive if they have the same sign. We may also rewrite $\bv{F}$ in terms of the \eax{electric field} $\bv{E}$ defined by $\bv{F} = Q \bv{E}$. Thus, over a set of source charges $q_{i}$, the electric field at position $\bv{r}$ is given by
\begin{align}
    \bv{E}(\bv{r}) = \frac{1}{4 \pi \epsilon_{0}} \sum_{i} \frac{q_{i}}{\curs{r}_{i}^{2}} \hat{\bcurs{r}}_{i},
\end{align}
where $\bcurs{r}_{i}$ is defined similarly as above for each source charge. For a continuous distribution of source charges, typically represented by a linear density $\lambda$, surface density $\sigma$ or volume density $\rho$, an integral appropriately replaces the summation. In this case, we consider $\di{q} = \lambda \; \di{l}$, $\di{q} = \sigma \; \di{A}$ or $\di{q} = \rho \; \di{V}$ as the infinitesimal charge elements, and integrate over the entire distribution to find the electric field at the point of interest. For example, for a volume charge distribution, the electric field is given by
\begin{align}
    \bv{E}(\bv{r}) = \frac{1}{4 \pi \epsilon_{0}} \int \frac{\di{q}}{\curs{r}^{2}} \hat{\bcurs{r}} = \frac{1}{4 \pi \epsilon_{0}} \int \frac{\rho(\bv{r}')}{\curs{r}^{2}} \hat{\bcurs{r}} \; \di{V}'.
\end{align}

\section{Divergence of $\bv{E}$: Gauss's Law and Electric Flux}

The electric field $\bv{E}(\bv{r})$ is a \eax{vector field}, meaning that at every point in space, it assigns a vector quantity. We can represent this field by assigning an `arrow' at each point in space, where the direction of the arrow indicates the direction of the electric field vector, and the length of the arrow represents the magnitude of the field at that point. This gets messy. To visualise the electric field, we often use \eax{field lines}, which are imaginary lines that represent the direction of the electric field. The density of these lines indicates the strength of the field; closer lines correspond to a stronger field. Field lines originate from positive charges and terminate on negative charges, providing a visual representation of how the electric field behaves in space.

\begin{figure}[h]
    \centering
    \includegraphics[width=0.4\textwidth]{images/field_lines.png}
    \caption{The electric field lines around a positive and a negative point charge.}
    \label{fig:field_lines}
\end{figure}

Associated with the electric field is the concept of \eax{electric flux}. Electric flux quantifies the amount of electric field passing through a given surface $S$. It is defined as
\begin{align}
    \Phi_{E} = \int_{S} \bv{E} \cdot \di{\bv{a}},
\end{align}
where $\di{\bv{a}}$ is an infinitesimal area element on the surface $S$, and the dot product $\bv{E} \cdot \di{\bv{a}}$ represents the component of the electric field passing through that area element. For example, let us consider a point charge $Q$ at the origin and a spherical surface of radius $R$ centred at the origin. We wish to find the electric flux through this spherical surface $S$. Converting to spherical coordinates, we have
\begin{align}
    \Phi_{E} = \oint_{S} \bv{E} \cdot \di{\bv{a}} = \frac{1}{4 \pi \epsilon_{0}} \int \frac{Q}{R^{2}} R^{2} \sin \theta \; \di{\theta} \; \di{\phi} = \frac{Q}{4 \pi \epsilon_{0}} \int_{0}^{\pi} \sin \theta \; \di{\theta} \int_{0}^{2 \pi} \di{\phi} = \frac{Q}{\epsilon_{0}}.
\end{align}
Here, $\oint$ represents a surface integral over a closed surface. \eax{Gauss's law} generalizes this result as
\begin{align}
    \Phi_{E} = \oint_{S} \bv{E} \cdot \di{\bv{a}} = \frac{Q_{\text{enc}}}{\epsilon_{0}},
\end{align}
where $Q_{\text{enc}}$ is the total charge enclosed within the surface $S$. Using the divergence theorem, we have
\begin{align}
    \int_{V} \nabla \cdot \bv{E} \; \di{V'} = \oint_{S} \bv{E} \cdot \di{\bv{a}} = \int_{V} \frac{\rho(\bv{r}')}{\epsilon_{0}} \; \di{V'}.
\end{align}
Since this holds for any arbitrary volume $V$, we obtain the differential form of Gauss's law:
\begin{align}
    \nabla \cdot \bv{E} = \frac{\rho}{\epsilon_{0}}.
\end{align}
\\
\textit{January 7th.}

In some sense, Coulomb's law is more fundamental than Gauss's law, despite both being equivalent. Coulomb's law follows from the inverse-square nature of the electric field, which in turn arises from the three-dimensional nature of space. If the electric field were to fall off with distance $r$ in a different manner, say as $1/r^{3}$, then Gauss's law would not hold in its current form. 

Let us now explicitly derive the \eax{differential form of Gauss's law}. Note that we had

\begin{align}
    \bv{E}(\bv{r}) = \frac{1}{4 \pi \epsilon_{0}} \int \frac{\rho(\bv{r}')}{\curs{r}^{2}} \hat{\bcurs{r}} \; \di{V}'.
\end{align}
Taking the divergence of both sides, we have
\begin{align}
    \nabla \cdot \bv{E} = \frac{1}{4 \pi \epsilon_{0}} \int \rho(\bv{r}') \nabla \cdot \left( \frac{\hat{\bcurs{r}}}{\curs{r}^{2}} \right) \; \di{V}'
\end{align}
since the divergence operator acts on $\bv{r}$, not $\bv{r}'$ (a fixed quantity). Using the cartesian coordinates now gets messy quickly, so spherical coordinates are preferred; in these coordinates, the divergence of any vector field $\bv{A} = A_{r} \hat{\bv{r}} + A_{\theta} \hat{\bv{\theta}} + A_{\phi} \hat{\bv{\phi}}$ is given by
\begin{align}
    \nabla \cdot \bv{A} = \frac{1}{r^{2}} \frac{\partial}{\partial r} (r^{2} A_{r}) + \frac{1}{r \sin \theta} \frac{\partial}{\partial \theta} (A_{\theta} \sin \theta) + \frac{1}{r \sin \theta} \frac{\partial}{\partial \phi} A_{\phi}.
\end{align}
Applying this to $\bv{A} = \hat{\bv{r}} / \abs{r}^{2}$, we find that $\nabla \cdot \left( \hat{\bv{r}} / \abs{r}^{2} \right) = \frac{1}{r^{2}} \frac{\partial}{\partial r} (r^{2} \cdot \frac{1}{r^{2}}) = 0$ for $\bv{r} \neq 0$. Let us integrate this over the volume of a sphere of radius $R$ centred at the origin:
\begin{align}
    \int_{V} \nabla \cdot \bv{A} \; \di{V'} = \oint_{S} \bv{A} \cdot \di{\bv{a}} = \oint_{S} \frac{\hat{\bv{r}}}{r^{2}} \cdot d \bv{a} = \int_{0}^{2\pi} \int_{0}^{\pi} \frac{1}{R^{2}} R^{2} \sin \theta \; \di{\theta} \; \di{\phi} = 4\pi.
\end{align}
This is a \textit{contradictory} result, since we found that the divergence is zero everywhere except at the origin, so the integration should be zero. The resolution to this was provided by Dirac, who introduced the concept of the \eax{Dirac delta function} $\delta(x)$. This is not a function in the traditional sense, but rather something known as a distribution. It is defined such that it is zero everywhere except at $x = 0$, where it is \textit{infinitely} large, and its integral over the entire real line is equal to one:
\begin{align}
    \delta(x) = \begin{cases}
        0 &\text{if } x \neq 0, \\
        \infty &\text{if } x = 0,
    \end{cases} \quad \text{and} \quad \int_{-\infty}^{\infty} \delta(x) \; \di{x} = 1.
\end{align}
One may also note that $\int_{-\infty}^{\infty} \delta(x - a) \; \di{x} = 1$ for any real number $a$. If $f(x)$ is a `well-behaved' function, then
\begin{align}
    \int_{-\infty}^{\infty} f(x) \delta(x - a) \; \di{x} = f(a).
\end{align}
The Dirac delta function can be generalised to higher dimensions; in three dimensions, it is defined as $\delta^{3}(\bv{r}) = \delta(x) \delta(y) \delta(z)$. Using this, we can write
\begin{align}
    \nabla \cdot \left( \frac{\hat{\bv{r}}}{r^{2}} \right) = 4 \pi \delta^{3}(\bv{r}).
\end{align}
This generalizes to
\begin{align}
    \nabla \cdot \left( \frac{\hat{\bcurs{r}}}{\curs{r}^{2}} \right) = 4 \pi \delta^{3}(\bcurs{r})
\end{align}
where the divergence is taken with respect to $\bv{r}$. Thus, the divergence of the electric field becomes
\begin{align}
    \nabla \cdot \bv{E} = \frac{1}{4 \pi \epsilon_{0}} \int \rho(\bv{r}') \nabla \cdot \left( \frac{\hat{\bcurs{r}}}{\curs{r}^{2}} \right) \; \di{V}' = \frac{1}{4 \pi \epsilon_{0}} \int \rho(\bv{r}') 4 \pi \delta^{3}(\bcurs{r}) \; \di{V}' = \frac{1}{\epsilon_{0}} \int \rho(\bv{r}') \delta^{3}(\bv{r} - \bv{r}') \; \di{V}' = \frac{\rho(\bv{r})}{\epsilon_{0}}.
\end{align}
This recovers Gauss's law in differential form.

\subsubsection{Common charge densities}

The charge distributions we encounter in electrostatics are often highly symmetric, allowing us to exploit this symmetry to simplify calculations. 

\begin{enumerate}[label=\arabic*.]
    \item \textit{Point charge}: A point charge is an idealized model of a charged particle with negligible size. The charge density for a point charge $q$ located at the origin is simply given by
    \begin{align}
        \rho(\bv{r}) = q \, \delta^{3}(\bv{r}).
    \end{align}

    \item \textit{Dipole}: A dipole consists of two equal and opposite point charges separated by a small distance. For a point charge $+q$ at the origin and a point charge $-q$ at position $\bv{a}$, the charge density is given by
    \begin{align}
        \rho(\bv{r}) = q \, \delta^{3}(\bv{r}) - q \, \delta^{3}(\bv{r} - \bv{a}).
    \end{align}

    \item \textit{Hollow sphere}: Consider a uniformly charged thin spherical shell of radius $R$ and total charge $Q$. The surface charge density $\sigma$ is given by $\sigma = \frac{Q}{4 \pi R^{2}}$. The volume charge density is then
    \begin{align}
        \rho(\bv{r}) = \sigma \, \delta(r - R) = \frac{Q}{4 \pi R^{2}} \, \delta(r - R).
    \end{align}
    To derive this, we note that $\rho(\bv{r})$ must be singular at the surface of the sphere, so it must be proportional to $\delta(r - R)$. To find the proportionality constant, we integrate over all space to ensure the total charge is $Q$:
    \begin{align}
        Q = \int \rho(\bv{r}) \; \di{V'} = 4 \pi \int_{0}^{\infty} r^{2} \, \rho(r) \, \di{r} = 4 \pi k \int_{0}^{\infty} r^{2} \, \delta(r - R) \, \di{r} = 4 \pi k R^{2} \implies k = \frac{Q}{4 \pi R^{2}}. 
    \end{align}
\end{enumerate}

\section{Curl of $\bv{E}$: Electric Potential}

Having discussed the divergence of the electric field, we now turn our attention to its curl; that is, we wish to compute $\nabla \times \bv{E}$. Consider the case of a point charge $q$ at the origin. Stokes' theorem gives us
\begin{align}
    \int_{S} (\nabla \times \bv{E}) \cdot \di{\bv{a}} = \oint_{C} \bv{E} \cdot \di{\bv{l}}.
\end{align}
Plugging in the expression for $\bv{E}$, and using the fact that $\di{\bv{l}} = \di{r} \hat{\bv{r}} + r \, \di{\theta} \hat{\bv{\theta}} + r \sin \theta \, \di{\phi} \hat{\bv{\phi}}$, let us integrate the line integral from $\bv{a}$ to $\bv{b}$ along some arbitrary path $C$:
\begin{align}
    \int_{C} \bv{E} \cdot \di{\bv{l}} = \frac{q}{4 \pi \epsilon_{0}} \int_{\bv{a}}^{\bv{b}} \frac{1}{r^{2}} \hat{\bv{r}} \cdot \di{\bv{l}} = \frac{q}{4 \pi \epsilon_{0}} \int_{r_{a}}^{r_{b}} \frac{1}{r^{2}} \, \di{r} = \frac{q}{4 \pi \epsilon_{0}} \left(\frac{1}{r_{a}} - \frac{1}{r_{b}}\right).
\end{align}
This result is independent of the path taken between points $\bv{a}$ and $\bv{b}$. Therefore, for any closed loop $C$, the line integral evaluates to zero, that is, $\oint \bv{E} \cdot \di{\bv{l}} = 0$. Since this holds for any arbitrary surface $S$ bounded by the loop $C$, we conclude that
\begin{align}
    \nabla \times \bv{E} = \bv{0}.
\end{align}
In fact, this result holds for any charge distribution and not just point charges. Note that this is a static situation; if charges were moving, then a time-varying magnetic field would be induced, leading to a non-zero curl of the electric field which we shall study later. We can also derive this result via computing the curl directly. Consider again a point charge $q$ at the origin. In cartesian coordinates, the electric field is given by
\begin{align}
    \bv{E}(\bv{r}) = \frac{q}{4 \pi \epsilon_{0}} \frac{\hat{\bv{r}}}{r^{3}} = \frac{q}{4 \pi \epsilon_{0}} \left( \frac{x \hat{\bv{x}} + y \hat{\bv{y}} + z \hat{\bv{z}}}{(x^{2} + y^{2} + z^{2})^{3/2}} \right).
\end{align}
Calculating the curl in cartesian coordinates is relatively straightforward, albeit a bit tedious via the determinant method:
\begin{align}
    \nabla \times \bv{E} = \begin{vmatrix}
        \hat{\bv{x}} & \hat{\bv{y}} & \hat{\bv{z}} \\
        \frac{\partial}{\partial x} & \frac{\partial}{\partial y} & \frac{\partial}{\partial z} \\
        E_{x} & E_{y} & E_{z}
    \end{vmatrix}.
\end{align}
With this, one easily finds that $\nabla \times \bv{E} = \bv{0}$ for $\bv{r} \neq 0$. The singularity at the origin can be handled similarly as before using the Dirac delta function. Again, the result can be generalised to any (static) charge distribution by integrating over the entire distribution.

Since $\int \bv{E} \cdot \di{\bv{l}}$ was found to be independent of the path taken, we are fit to define a scalar function $V(\bv{r})$ known as the \eax{electric potential} such that
\begin{align}
    V(\bv{r}) = - \int_{O}^{\bv{r}} \bv{E} \cdot \di{\bv{l}},
\end{align}
where $O$ is some standard point, often taken to be at infinity. This choice of reference point is arbitrary, as only the \eax{potential difference} between two points is physically meaningful:
\begin{align}
    V(\bv{b}) - V(\bv{a}) = - \int_{O}^{\bv{b}} \bv{E} \cdot \di{\bv{l}} + \int_{O}^{\bv{a}} \bv{E} \cdot \di{\bv{l}} = - \int_{\bv{a}}^{\bv{b}} \bv{E} \cdot \di{\bv{l}}.
\end{align}
What is also true is that
\begin{align}
    V(\bv{b}) - V(\bv{a}) = - \int_{\bv{a}}^{\bv{b}} \bv{E} \cdot \di{\bv{l}} \implies \bv{E} = - \nabla V = - \left( \frac{\partial V}{\partial x} \hat{\bv{x}} + \frac{\partial V}{\partial y} \hat{\bv{y}} + \frac{\partial V}{\partial z} \hat{\bv{z}} \right).
\end{align}
The advantage of working with the electric potential $V$ instead of the electric field $\bv{E}$ is that $V$ is a scalar function, making calculations often simpler. Note that this also implies $V$ is arbitrary up to an additive constant, since adding a constant to $V$ does not change $\bv{E}$. The principle of superposition also holds for electric potentials; the total potential due to a set of source charges is simply the algebraic sum of the potentials due to each individual charge. Thus, we have derived two fundamental properties of the electrostatic field:
\begin{align}
    \nabla \cdot \bv{E} = \frac{\rho}{\epsilon_{0}}, \quad \text{and} \quad \nabla \times \bv{E} = \bv{0}.
\end{align}
These are differential equations that govern the behaviour of the electrostatic field $\bv{E}$ in the presence of a charge distribution $\rho$. Using the relation $\bv{E} = - \nabla V$, we can rewrite the differential equations in terms of the electric potential $V$:
\begin{align}
    \nabla^{2} V = - \frac{\rho}{\epsilon_{0}}, \quad \text{and} \quad \nabla \times \nabla V = \bv{0}.
\end{align}
Here, $\nabla^{2} = \nabla \cdot \nabla$ is the Laplacian operator. The first equation is known as \eax{Poisson's equation}, while the second is an identity that holds for any scalar function. If there are no charges present in a region, that is, $\rho = 0$, then Poisson's equation reduces to \eax{Laplace's equation}, $\nabla^{2} V = 0$.
\\ \\
\textit{January 12th.}

Suppose $\bv{E} \sim \frac{1}{r^{3}} \hat{\bv{r}}$, that is, the electric field falls off as the cube of the distance from a point charge. Does it still hold true that $\nabla \times \bv{E} = \bv{0}$? Yes, since if a force only depends on the vector joining two points and not on the path taken, then the force must be conservative, and its curl must be zero, giving $q \nabla \times \bv{E} = \bv{0}$. However, Gauss's law would not hold in this case.

Let us take a look at how the potential $V$ takes form. Start from a point charge $q$ at the origin. Then
\begin{align}
    V = - \int_{\infty}^{\bv{r}} \bv{E} \cdot \di{\bv{l}} = - \int_{\infty}^{\bv{r}} \frac{q}{4 \pi \epsilon_{0}} \frac{1}{r^{2}} \hat{\bv{r}} \cdot \hat{\bv{r}} \, \di{r} = \frac{q}{4 \pi \epsilon_{0} r}.
\end{align}
Thus, the potentia due to a point charge at $\bv{r}'$ is given by
\begin{align}
    V(\bv{r}) = \frac{q}{4 \pi \epsilon_{0} \curs{r}}.
\end{align}
For a set of point charges $q_{i}$ at positions $\bv{r}_{i}'$, the potential is given by
\begin{align}
    V(\bv{r}) = \frac{1}{4 \pi \epsilon_{0}} \sum_{i} \frac{q_{i}}{\curs{r}_{i}}.
\end{align}
For a continuous distribution of charge with volume density $\rho(\bv{r}')$, the potential is given by
\begin{align}
    V(\bv{r}) = \frac{1}{4 \pi \epsilon_{0}} \int \frac{\di{q}}{\curs{r}} = \frac{1}{4 \pi \epsilon_{0}} \int \frac{\rho(\bv{r}')}{\curs{r}} \; \di{V}'.
\end{align}
This is the formal solution to Poisson's equation.

One could compute the potential due to a uniformly charged spherical shell, but it is easier to use Gauss's law to find the electric field first, and then integrate to find the potential. Let us look at the case of a plate; consider a uniformly charged plate if radius $R$ and surface charge density $\sigma$. The potential at a point along the axis of the plate at a distance $z$ from its centre is given by
\begin{align}
    V(z) = \frac{1}{4 \pi \epsilon_{0}} \int \frac{\sigma(\bv{r}')}{\curs{r}} \; \di{A}' = \frac{\sigma}{4 \pi \epsilon_{0}} \int_{0}^{2 \pi} \int_{0}^{R} \frac{r' \, \di{r}' \, \di{\phi}'}{\sqrt{r'^{2} + z^{2}}} = \frac{\sigma}{2 \epsilon_{0}} \left( \sqrt{R^{2} + z^{2}} - z \right).
\end{align}
Thus, the electric field is given by
\begin{align}
    E_{z} = - \frac{\di{V}}{\di{z}} = \frac{\sigma}{2 \epsilon_{0}} \left( 1 - \frac{z}{\sqrt{R^{2} + z^{2}}} \right).
\end{align}

\subsection{Electrostatic Boundary Conditions}

Let us look at the above example when the plate is infinite; consider an infinite plane with uniform surface charge density $\sigma$. By symmetry, the electric field must point directly away from the plane (if $\sigma > 0$) or towards the plane (if $\sigma < 0$). Using Gauss's law, we can find the magnitude of the electric field. Consider a cylindrical Gaussian surface that straddles the plane, with its flat faces parallel to the plane. The flux through the curved surface is zero since the electric field is perpendicular to it. The flux through the two flat faces (cyclindrical surface) is given by
\begin{align}
    \Phi_{E} = \oint \bv{E} \cdot \di{\bv{a}} = \frac{Q_{\text{enc}}}{\epsilon_{0}} \implies 2 E A = \frac{\sigma A}{\epsilon_{0}} \implies E = \frac{\sigma}{2 \epsilon_{0}}.
\end{align}
Note that there is a discontinuity in the electric field as we cross the plane. Just above the plane, the electric field is $\bv{E} = \frac{\sigma}{2 \epsilon_{0}} \hat{\bv{n}}$, while just below the plane, it is $\bv{E} = - \frac{\sigma}{2 \epsilon_{0}} \hat{\bv{n}}$, where $\hat{\bv{n}}$ is the unit normal vector pointing away from the plane.
\\

Now consider an arbitrary surface with a (not necessarily uniform) surface charge density $\sigma$. Consider a similar cylindrical Gaussian surface that straddles the surface; this cylinder is infinitesimal, with height $2 \epsilon$ and cross-sectional area $A$. Here, we have
\begin{align}
    \oint \bv{E} \cdot \di{\bv{a}} = \frac{\sigma A}{\epsilon_{0}} \implies (E_{\perp}^{\text{above}} - E_{\perp}^{\text{below}}) A = \frac{\sigma A}{\epsilon_{0}} \implies E_{\perp}^{\text{above}} - E_{\perp}^{\text{below}} = \frac{\sigma}{\epsilon_{0}},
\end{align}
implying that the normal component of the electric field has a discontinuity across a surface charge. Here, $E_{\perp}^{\text{above}}$ and $E_{\perp}^{\text{below}}$ are the normal components of the electric field just above and just below the surface, respectively.

Through this surface, condier a small rectangular loop that pierces the surface; the loop has two sides parallel to the surface of length $l$ and two sides perpendicular to the surface of height $\epsilon$. We have
\begin{align}
    \oint \bv{E} \cdot \di{\bv{l}} = 0 \implies E_{\parallel}^{\text{above}} l - E_{\parallel}^{\text{below}} l = 0 \implies E_{\parallel}^{\text{above}} = E_{\parallel}^{\text{below}},
\end{align}
implying that the tangential component of the electric field is continuous across the surface. Thus, we can conclude that
\begin{align}
    \bv{E}_{\text{above}} - \bv{E}_{\text{below}} = \frac{\sigma}{\epsilon_{0}} \hat{\bv{n}}.
\end{align}
We discuss what happens to the potential $V$; since $\bv{E} = - \nabla V$, we have
\begin{align}
    - \int_{\bv{a}}^{\bv{b}} \bv{E}_{\text{above}} \cdot \di{\bv{l}} + \int_{\bv{a}}^{\bv{b}} \bv{E}_{\text{below}} \cdot \di{\bv{l}} = -\frac{\sigma}{\epsilon_{0}} \int_{\bv{a}}^{\bv{b}} \hat{\bv{n}} \cdot \di{\bv{l}} = -\frac{\sigma}{\epsilon_{0}} \epsilon + \frac{\sigma}{\epsilon_{0}} \epsilon = 0,
\end{align}
implying that
\begin{align}
    V_{\text{above}} - V_{\text{below}} = 0.
\end{align}
Thus, the electric potential is continuous across a surface charge, even though the electric field may be discontinuous. If we replace $\bv{E}$ with $- \nabla V$ in the earlier boundary condition for $\bv{E}$, we find that
\begin{align}
    \hat{\bv{n}} \cdot \left( -\nabla V_{\text{above}} + \nabla V_{\text{below}} \right) = \frac{\sigma}{\epsilon_{0}} \hat{\bv{n}} \cdot \hat{\bv{n}} \implies \left(- \frac{\partial V}{\partial n} \right)_{\text{above}} + \left( \frac{\partial V}{\partial n} \right)_{\text{below}} = \frac{\sigma}{\epsilon_{0}}.
\end{align}

\section{Work and Energy}

Recall that for a conservative force field, the work done in moving a particle from point $\bv{a}$ to point $\bv{b}$ is given by
\begin{align}
    W_{\bv{a} \to \bv{b}} = \int_{\bv{a}}^{\bv{b}} \bv{F} \cdot \di{\bv{l}}.
\end{align}
Since the electrostatic force is conservative, for a test charge $Q$ in an electric field $\bv{E}$, the work done by an external agent in moving the charge from point $\bv{a}$ to point $\bv{b}$ is given by
\begin{align}
    W_{\bv{a} \to \bv{b}} = -\int_{\bv{a}}^{\bv{b}} Q \bv{E} \cdot \di{\bv{l}} = -Q \int_{\bv{a}}^{\bv{b}} \bv{E} \cdot \di{\bv{l}} = Q (V(\bv{b}) - V(\bv{a})).
\end{align}
Thus, we define
\begin{align}
    W = Q V.
\end{align}

Let us take up an example to illustrate this work done; let there be $n$ point charges $q_{1}, q_{2}, \ldots, q_{n}$ in some configuration. We wish to find the work done in bringing these charges in from infinity to their respective positions. The work done in bringing in the first charge $q_{1}$ is zero, since there are no other charges present. The work done in bringing in the second charge $q_{2}$ is given by
\begin{align}
    W_{2} = \frac{1}{4 \pi \epsilon_{0}} \frac{q_{1} q_{2}}{\curs{r}_{12}}.
\end{align}
Here, $\curs{r}_{12}$ is the distance between charges $q_{1}$ and $q_{2}$. The work done in bringing in the third charge $q_{3}$ is given by
\begin{align}
    W_{3} = \frac{1}{4 \pi \epsilon_{0}} \left( \frac{q_{1} q_{3}}{\curs{r}_{13}} + \frac{q_{2} q_{3}}{\curs{r}_{23}} \right).
\end{align}
Continuing in this manner, the total work done in assembling the configuration of $n$ charges is given by
\begin{align}
    W = \frac{1}{4 \pi \epsilon_{0}} \sum_{i = 1}^{n} \sum_{1 \leq j < i} \frac{q_{i} q_{j}}{\curs{r}_{ij}} = \frac{1}{8 \pi \epsilon_{0}} \sum_{i \neq j} \frac{q_{i} q_{j}}{\curs{r}_{ij}}.
\end{align}
To write in terms of the electric potential, we have
\begin{align}
    W = \frac{1}{2} \sum_{i} q_{i} V(\bv{r}_{i}), \quad \text{where} \quad V(\bv{r}_{i}) = \sum_{j \neq i} \frac{1}{4 \pi \epsilon_{0}} \frac{q_{j}}{\curs{r}_{ij}}.
\end{align}
For a continuous charge distribution,
\begin{align}
    W = \frac{1}{2} \int \rho(\bv{r}) V(\bv{r}) \; \di{V'}.
\end{align}
From Gauss's law,
\begin{align}
    W = \frac{\epsilon_{0}}{2} \int (\nabla \cdot \bv{E}) V \; \di{V'}.
\end{align}
Here, we make use of the vector identity
\begin{align}
    \nabla \cdot (f \bv{A}) = (\nabla \cdot \bv{A}) f + \bv{A} \cdot (\nabla f).
\end{align}
This identity will be used throughout the course. Using this, we have
\begin{align}
    W = \frac{\epsilon_{0}}{2} \int \nabla \cdot (V \bv{E}) \; \di{V'} - \frac{\epsilon_{0}}{2} \int \bv{E} \cdot (\nabla V) \; \di{V'} = \frac{\epsilon_{0}}{2} \oint V \bv{E} \cdot \di{\bv{a}} + \frac{\epsilon_{0}}{2} \int \bv{E} \cdot \bv{E} \; \di{V'}.
\end{align}
The surface integral vanishes if we take the surface to be at infinity, since both $V\bv{E}$ will fall off sufficiently fast. Thus, we have
\begin{align}
    W = \frac{\epsilon_{0}}{2} \int E^{2} \; \di{V'}.
\end{align}

\noindent \textit{January 14th.}

Here, $W$ may also be interpreted as the energy of the charge distribution. Note that the energy density (energy per unit volume) of the electrostatic field is given by
\begin{align}
    \frac{\epsilon_{0}}{2} E^{2}.
\end{align}
We have two expressions for the work  done as $\frac{1}{2} \epsilon_{0} \int E^{2} \; \di{V'}$ and $\frac{1}{2} \sum_{i} q_{i} V(\bv{r}_{i})$. The first of these expressions is always positive, while the second expression can be negative if there are opposite charges present. This is due to the fact that in the second expression, charges are taken as given; they do not take into account the energy required to assemble the charges themselves. On the contrary, in the first expression, $\rho(\bv{r})$ is treated as a continuous distribution, and the self-energy of assembling the charge distribution is included.

Another important feature is that since $W \propto E^{2}$, the energy does not obey the principle of superposition. That is, if we have two electric fields $\bv{E}_{1}$ and $\bv{E}_{2}$ due to two different charge distributions, then the total energy is not given by the sum of the individual energies but rather as
\begin{align}
    W = \frac{\epsilon_{0}}{2} \int (\bv{E}_{1} + \bv{E}_{2})^{2} \; \di{V'} = \frac{\epsilon_{0}}{2} \int (E_{1}^{2} + E_{2}^{2} + 2 \bv{E}_{1} \cdot \bv{E}_{2}) \; \di{V'}.
\end{align}


\subsection{Conductors and Induced Charges}

The property of a \eax{conductor} is that the electrons (and the positive holes) are free to move within the material. When a conductor is placed in an external electric field, the free charges rearrange themselves in such a way that the electric field inside the conductor is zero. This is because if there were a non-zero electric field inside the conductor, it would exert a force on the free charges, causing them to move until they reach an equilibrium state where the internal electric field is nullified. Thus, in electrostatic equilibrium, the electric field inside a conductor is zero.

\newpage

\begin{figure}[h]
    \centering
    \includegraphics[width=0.4\textwidth]{images/conductor_in_field.jpg}
    \caption{\centering A conductor placed in an external electric field. Free charges within the conductor rearrange themselves to cancel the field inside the conductor.}
    \label{fig:conductor_in_field}
\end{figure}

That is, if the field applied was $\bv{E}_{0}$, then the induced field $\bv{E}_{\text{ind}}$ within the conductor is such that $\bv{E}_{0} + \bv{E}_{\text{ind}} = \bv{0}$. Thus, $\bv{E} = \bv{0}$ within a conductor. This also implies that the charge density $\rho$ is also zero within the conductor, since from Gauss's law, $\nabla \cdot \bv{E} = \frac{\rho}{\epsilon_{0}}$. Moreover, a conductor is an equipotential region; that is, the electric potential $V$ is constant throughout the conductor. This is because if there were a potential difference between two points within the conductor, it would create an electric field that would cause charges to move, contradicting the assumption of electrostatic equilibrium. In mathematical terms,
\begin{align}
    V = - \int \bv{E} \cdot \di{\bv{l}} = V(\bv{b}) - V(\bv{a}) = 0 \implies V(\bv{a}) = V(\bv{b}).
\end{align}
Another property worth noting is that the electric field $\bv{E}$ just outside the surface of a conductor must be perpendicular to the surface: we know that $V$ is constant on the surface. If we take $\di{\bv{r}}$ to be a small displacement along the surface, then
\begin{align}
    \nabla V \cdot \di{\bv{r}} = 0 \implies \nabla V \perp \di{\bv{r}}.
\end{align}


Now suppose we are dealing with a cavity within a conductor, as shown in Figure \ref{fig:conductor_cavity}. If there are no charges within the cavity, then the electric field within the cavity is zero. This is because if there were a non-zero electric field within the cavity, it would create a potential difference between points on the surface of the cavity. However, since the surface of the cavity is part of the conductor, and the conductor is an equipotential region, this would lead to a contradiction. Thus, the electric field within the cavity must also be zero if there are no charges present within it.

However, let us suppose a point charge $q$ is placed within the cavity. 


\begin{figure}[h]
    \centering
    \includegraphics[width=0.4\textwidth]{images/conductor_cavity.png}
    \caption{\centering A conductor placed in an external electric field. Free charges within the conductor rearrange themselves to cancel the field inside the conductor.}
    \label{fig:conductor_cavity}
\end{figure}


The presence of this charge induces a redistribution of charges on the inner surface of the cavity, resulting in an induced charge distribution that ensures the electric field within the conductor remains zero. If we denote $q_{\text{ind}}$ as the total induced charge on the inner surface of the cavity, we can determine its value using Gauss's law. Consider a Gaussian surface that lies just inside the conductor, enclosing the cavity and the point charge $q$. Since the electric field within the conductor is zero, the flux through this Gaussian surface is also zero. According to Gauss's law,
\begin{align}
    \oint \bv{E} \cdot \di{\bv{a}} = \frac{q + q_{\text{ind}}}{\epsilon_{0}} = 0 \implies q_{\text{ind}} = -q.
\end{align}
Thus, the total induced charge on the inner surface of the cavity is equal in magnitude but opposite in sign to the charge placed within the cavity. This induced charge distribution creates an electric field that cancels out the field produced by the charge $q$ within the conductor, maintaining the condition of electrostatic equilibrium. If the Gaussian surface is taken just outside the conductor, then the total enclosed charge is $q$, since the conductor as a whole is electrically neutral. Thus, the electric field outside the conductor is as if the conductor were not present at all, and is completely independent of the location or shape of the cavity within the conductor.

\subsubsection{Surface Charge and Force on a Conductor}
\textit{January 19th.}

Consider a conductor with a surface charge density $\sigma$. The electric field just outside the surface of the conductor is given by Gauss's law as
\begin{align}
    \bv{E} = \frac{\sigma}{\epsilon_{0}} \hat{\bv{n}}.
\end{align}
From here, one can see that
\begin{align}
    \sigma = -\epsilon_{0} \frac{\partial V}{\partial n}.
\end{align}
In the presence of an external electric field, the conductor experiences a force. The force per unit area (pressure) on the surface of the conductor due to the electric field is given by
\begin{align}
    \bv{f} = \sigma \bv{E}.
\end{align}
The questions is since the electric field is discontinuous across the surface of the conductor, which value of $\bv{E}$ should we use to compute the force? For a surface distribution, note that $\bv{E}_{\text{total}} = \bv{E}_{\text{patch}} + \bv{E}_{\text{other}}$, where $\bv{E}_{\text{patch}}$ is the field due to the surface charge patch itself, and $\bv{E}_{\text{other}}$ is the field due to all other charges. Thus, just above and below the surface of the conductor, we have
\begin{align}
    \bv{E}_{\text{total}}^{\text{above}} = \bv{E}_{\text{patch}}^{\text{above}} + \bv{E}_{\text{other}}, \quad \text{and} \quad \bv{E}_{\text{total}}^{\text{below}} = \bv{E}_{\text{patch}}^{\text{below}} + \bv{E}_{\text{other}}.
\end{align}
Noting that $\bv{E}_{\text{patch}}^{\text{above}} = \frac{1}{2\epsilon_{0}} \sigma \hat{\bv{n}}$ and $\bv{E}_{\text{patch}}^{\text{below}} = - \frac{1}{2\epsilon_{0}} \sigma \hat{\bv{n}}$, we have
\begin{align}
    \bv{E}_{\text{other}} = \frac{1}{2} \left( \bv{E}_{\text{above}} + \bv{E}_{\text{below}} \right).
\end{align}
This argument shows that the effective electric field that exerts a force on the surface charge is the average of the fields just above and just below the surface; this argument applies to any surface charge distribution. So, since $\bv{E}_{\text{below}} = \bv{0}$ for a conductor, we have
\begin{align}
    \bv{f} = \sigma \left( \frac{1}{2} \bv{E}_{\text{above}} \right) = \frac{1}{2} \frac{\sigma^{2}}{\epsilon_{0}} \hat{\bv{n}}.
\end{align}
The force per unit area on the surface of a conductor due to the electric field is given by
\begin{align}
    P = \frac{1}{2} \frac{\sigma^{2}}{\epsilon_{0}} = \frac{1}{2} \epsilon_{0} E^{2}.
\end{align}
Let us now focus on Laplace's equation, $\nabla^{2} V = 0$. This is a second-order partial differential equation. If this equation were one-dimensional, then
\begin{align}
    \frac{\di^{2} V}{\di{x}^{2}} = 0 \implies V(x) = mx + c,
\end{align}
where $m$ and $c$ are constants. To determine these constants, we need two boundary conditions, for example, the values of $V$ at two different points. In such a case, $V(x)$ is the average of $V(x+a)$ and $V(x-a)$:
\begin{align}
    V(x) = \frac{1}{2} \left( V(x+a) + V(x-a) \right).
\end{align}
This solution has no local maxima or minima; the maximum and minimum values of $V$ occur at the boundaries. 