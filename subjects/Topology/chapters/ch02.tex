\chapter{MAPPINGS}

\section{Continuous Functions}
Recall the definition of a continuous function on $\R$; a function $f :\R \to \R$ was said to be continuous at a point $x \in \R$ if for every $\varepsilon > 0$, there exists a $\delta > 0$ such that
\begin{align}
    \abs{f(x) - f(y)} < \varepsilon \text{ whenever } \abs{x - y} < \delta.
\end{align}

We can abstract this definition to metric spaces as follows.
\begin{definition}
    A function $f : (X,d_{X}) \to (Y,d_{Y})$ between two metric spaces is said to be \eax{continuous} at a point $x \in X$ if for every $\varepsilon > 0$, there exists a $\delta > 0$ such that
    \begin{align}
        d_{Y}(f(x), f(y)) < \varepsilon \text{ whenever } d_{X}(x,y) < \delta.
    \end{align}
    It is continuous at every point of $X$ if it is continuous at each $x \in X$.
\end{definition}

Also recall that a function $f : \R \to \R$ was continuous at $x \in \R$ if and only if for every open set $V \subseteq \R$ containing $f(x)$, the preimage $f^{-1}(V)$ is an open set in $\R$ containing $x$. We can use this characterization to define continuity in topological spaces.

\begin{definition}
    A function $f : (X,\tau_{X}) \to (Y,\tau_{Y})$ between two topological spaces is said to be \eax{continuous} at a point $x \in X$ if for every open set $V \in \tau_{Y}$ containing $f(x)$, the preimage $f^{-1}(V)$ is an open set in $\tau_{X}$ containing $x$. That is,
    \begin{align}
        \forall V \in \tau_{Y}, f(x) \in V \implies x \in f^{-1}(V) \in \tau_{X}.
    \end{align}
    It is continuous at every point of $X$ if it is continuous at each $x \in X$. That is,
    \begin{align}
        \forall V \in \tau_{Y} \implies f^{-1}(V) \in \tau_{X}.
    \end{align}
\end{definition}

The above statement can be made more tight via bases: $f$ is continuous if and only if the preimage of every basis element of $Y$ is an open set in $X$.

\begin{lemma}
    A function $f: (X,d_{X}) \to (Y,d_{Y})$ between two metric spaces is continuous if and only if $f:(X,\tau_{X}) \to (Y,\tau_{Y})$ is continuous, where $\tau_{X}$ and $\tau_{Y}$ are the topologies induced by the metrics $d_{X}$ and $d_{Y}$ respectively.
\end{lemma}

\begin{proof}
    Suppose $f: (X,d_{X}) \to (Y,d_{Y})$ is continuous. Let $V \in \tau_{Y}$ be an open set in $Y$. Then, for every $y \in V$, there exists an $\varepsilon_{y} > 0$ such that $B_{Y}(y,\varepsilon_{y}) \subseteq V$. Since $f$ is continuous, for every $x \in f^{-1}(V)$, there exists a $\delta_{x} > 0$ such that
    \begin{align}
        f(B_{X}(x,\delta_{x})) \subseteq B_{Y}(f(x),\varepsilon_{f(x)}) \subseteq V.
    \end{align}
    Thus, $B_{X}(x,\delta_{x}) \subseteq f^{-1}(V)$. Hence, $f^{-1}(V)$ is open in $X$ and so $f:(X,\tau_{X}) \to (Y,\tau_{Y})$ is continuous.

    For the converse, suppose $f:(X,\tau_{X}) \to (Y,\tau_{Y})$ is continuous. That is, for every open set $V \in \tau_{Y}$, the preimage $f^{-1}(V)$ is an open set in $\tau_{X}$. Let $\varepsilon > 0$ be given. Consider the open ball $B_{Y}(f(x),\varepsilon) \in \tau_{Y}$. Since $f$ is continuous, $f^{-1}(B_{Y}(f(x),\varepsilon))$ is open in $X$ and contains $x$. Thus, there exists a $\delta > 0$ such that $B_{X}(x,\delta) \subseteq f^{-1}(B_{Y}(f(x),\varepsilon))$. Hence, for every $y \in B_{X}(x,\delta)$, we have
    \begin{align}
        f(y) \in B_{Y}(f(x),\varepsilon),
    \end{align}
    proving that $f: (X,d_{X}) \to (Y,d_{Y})$ is continuous.
\end{proof}



\begin{lemma}
    Let $(X,\tau)$ be a topological space and let $A \subseteq X$. Then the following hold.
    \begin{enumerate}[label=(\roman*)]
        \item For every sequence $(a_{n})_{n \in \N} \subseteq A$, if $a_{n} \to x$ in $X$, then $x \in \overline{A}$. That is, if $x \in U$ for some open set $U \in \tau$, there exists a $N(U) \equiv N \in \N$ such that $a_{n} \in A \cap U$ for all $n \geq N$.
        \item The converse holds if $X$ is a metric space.
        \item In general, the converse need not hold.
    \end{enumerate}
\end{lemma}
\begin{proof}
    We prove the second statement only. Suppose $x \in \overline{A}$. Then $B(x, \frac{1}{n}) \cap A \neq \emptyset$ for every $n \in \N$. Thus, we can choose $a_{n} \in B(x, \frac{1}{n}) \cap A$ for each $n \in \N$. Then, for every $\varepsilon > 0$, there exists a $N \equiv N(\varepsilon) \in \N$ such that $\frac{1}{N} < \varepsilon$. Thus, for every $n \geq N$, we have
    \begin{align}
        d(a_{n}, x) < \frac{1}{n} \leq \frac{1}{N} < \varepsilon,
    \end{align}
    proving that $a_{n} \to x$ in $X$.
\end{proof}

There is a third definition for metric spaces. A function $f:(X,d_{X}) \to (Y,d_{Y})$ between two metric spaces is said to be \eax{continuous} at a point $x \in X$ if for every sequence $(x_{n})_{n \in \N} \subseteq X$ such that $x_{n} \to x$ in $X$, we have $f(x_{n}) \to f(x)$ in $Y$. In fact, the open-set definition implies the sequential definition.

\begin{proof}
    Assume the open-set definition of continuity. Let $(x_{n})_{n \in \N} \subseteq X$ be a sequence such that $x_{n} \to x$ in $X$. Let $\varepsilon > 0$ be given. Consider the open ball $B_{Y}(f(x),\varepsilon) \in \tau_{Y}$. Since $f$ is continuous, $f^{-1}(B_{Y}(f(x),\varepsilon))$ is open in $X$ and contains $x$. Thus, there exists a $\delta > 0$ such that $B_{X}(x,\delta) \subseteq f^{-1}(B_{Y}(f(x),\varepsilon))$. Since $x_{n} \to x$ in $X$, there exists a $N \equiv N(\delta) \in \N$ such that for every $n \geq N$, we have
    \begin{align}
        x_{n} \in B_{X}(x,\delta) \implies f(x_{n}) \in B_{Y}(f(x),\varepsilon),
    \end{align}
    proving that $f(x_{n}) \to f(x)$ in $Y$.
\end{proof}

For metric spaces, the sequential definition also implies the open-set definition.

\begin{example}
    Recall $\R_{l}$, which was generated from the basis $\cB = \{[a,b) : a < b, a,b \in \R\}$. Consider the identity function $f : \R \to \R_{l}$, with $x \mapsto x$. Then $[0,1)$ is open in $\R_{l}$, but its preimage under $f$ is $[0,1)$, which is not open in $\R$.  
\end{example}


\begin{theorem}
    Let $f:(X,\tau_{X}) \to (Y,\tau_{Y})$ be a function between two topological spaces. Then the following are equivalent.
    \begin{enumerate}[label=(\roman*)]
        \item $f$ is (open-set) continuous.
        \item For every closed $C \subseteq Y$, the preimage $f^{-1}(C)$ is closed in $X$.
        \item For every $A \subseteq X$, we have $f(\overline{A}) \subseteq \overline{f(A)}$.
        \item For all $f(x) \in V$ with $V \in \tau_{Y}$, there exists $U \in \tau_{X}$ such that $x \in U$ and $f(U) \subseteq V$.
    \end{enumerate}
\end{theorem}

\textit{January 20th.}

\begin{proof}
    We first show that (i) and (ii) are equivalent. Suppose $C$ is closed in $Y$. Then, $Y \setminus C$ is open in $Y$. Since $f$ is continuous, the preimage $f^{-1}(Y \setminus C)$ is open in $X$. But,
    \begin{align}
        f^{-1}(Y \setminus C) = X \setminus f^{-1}(C).
    \end{align}
    Thus, $f^{-1}(C)$ is closed in $X$. One can follow the converse argument to show that (ii) implies (i).

    For (i) implies (iii), let $A \subseteq X$ and let $x \in \overline{A}$. Let $V$ be an open neighbourhood of $f(x)$ in $Y$. Since $f$ is continuous, $f^{-1}(V)$ is an open neighbourhood of $x$ in $X$. Thus, $f^{-1}(V) \cap A \neq \emptyset$. Let $a \in f^{-1}(V) \cap A$. Then, $f(a) \in V$ and $f(a) \in f(A)$. Since $V$ was arbitrary, we must have $V \cap f(A) \neq \emptyset$. Thus, $f(x) \in \overline{f(A)}$, proving that $f(\overline{A}) \subseteq \overline{f(A)}$. For (iii) implies (ii), let $V$ be a closed set in $Y$. Then
    \begin{align}
        f^{-1}(V) \subseteq X \implies f(\overline{f^{-1}(V)}) \subseteq \overline{f(f^{-1}(V))} \subseteq \overline{V} = V
    \end{align}
    which tells us $\overline{f^{-1}(V)} \subseteq f^{-1}(V)$, proving that $f^{-1}(V)$ is closed in $X$.

    For (i) implies (iv), let $f(x) \in V$ with $V \in \tau_{Y}$. Since $f$ is continuous, $f^{-1}(V)$ is open in $X$ and contains $x$. Thus, we can take $U = f^{-1}(V)$, proving (iv). For (iv) implies (i), let $V \in \tau_{Y}$. For every $x \in f^{-1}(V)$, we have $f(x) \in V$. By (iv), there exists an open set $U_{x} \in \tau_{X}$ such that $x \in U_{x}$ and $f(U_{x}) \subseteq V$. Thus, taking the union over all $x \in f^{-1}(V)$ gives $\bigcup_{x \in f^{-1}(V)} U_{x} = f^{-1}(V)$, proving that $f^{-1}(V)$ is open in $X$.
\end{proof}


\subsection{Rules of Continuous Functions}

Note that the constant function $f : X \to Y$ defined by $f(x) = y_{0}$ for some fixed $y_{0} \in Y$ is continuous. Also, the identity function $\id_{X} : X \to X$ defined by $\id_{X}(x) = x$ is continuous.

Let $X$ be a topological space, and $A$ be a subset of $X$ with the subspace topology. Then the inclusion map $i : A \hookrightarrow X$ defined by $i(a) = a$ is continuous. If $f: X \to Y$ and $g : Y \to Z$ are continuous functions between topological spaces, then the composition $g \circ f : X \to Z$ defined by $(g \circ f)(x) = g(f(x))$ is also continuous.

\begin{theorem}
    Let $f_{1}:Z \to X$ and $f_{2}:Z \to Y$ be functions between topological spaces. Then the function $f : Z \to X \times Y$ defined by $f(z) = (f_{1}(z), f_{2}(z))$ is continuous if and only if both $f_{1}$ and $f_{2}$ are continuous.
\end{theorem}
It is important to note that a function $f: X \to Y$ is continuousif and only if $f^{-1}(B)$ is open in $X$ for every basis element $B$ of $Y$.
\begin{proof}
    For the forward implication, it is enough to show that $f^{-1}(U \times V) = f_{1}^{-1}(U) \cap f_{2}^{-1}(V)$ is open in $Z$, where $U$ and $V$ are open sets in $X$ and $Y$ respectively. $f$ is continuous implies that $f^{-1}(U \times V)$ is open in $Z$. In particular, both $f_{1}^{-1}(U)$ and $f_{2}^{-1}(V)$ are open in $Z$, proving that both $f_{1}$ and $f_{2}$ are continuous.

    Conversely, suppose both $f_{1}$ and $f_{2}$ are continuous. Let $U \times V$ be a basis element of $X \times Y$, where $U$ and $V$ are open sets in $X$ and $Y$ respectively. By continuity of $f_{1}$ and $f_{2}$, both $f_{1}^{-1}(U)$ and $f_{2}^{-1}(V)$ are open in $Z$. Thus $f^{-1}(U \times V) = f_{1}^{-1}(U) \cap f_{2}^{-1}(V)$ is open in $Z$, proving that $f$ is continuous.
\end{proof}

\begin{lemma}
    Let $(X,d)$ be a metric space, and let $f: X \to \R$ be a continuous function. Suppose $f(x) \neq 0$ for some $x \in X$. Then there exists $r > 0$ such that $f(y) \neq 0$ for all $y \in B(x,r)$. Moreover, the map $g:B(x,r) \to \R$ defined by $g(y) = \frac{1}{f(y)}$ is continuous.
\end{lemma}

\begin{proof}
    Suppose such an $r$ does not exist. Then for every natural $n \in \N$, there exists a point $x_{n} \in B(x, \frac{1}{n})$ such that $f(x_{n}) = 0$. Thus, the sequence $(x_{n})_{n \in \N}$ converges to $x$ in $X$. Since $f$ is continuous, the sequence $(f(x_{n}))_{n \in \N}$ converges to $f(x)$ in $\R$. But $f(x_{n}) = 0$ for all $n \in \N$, so $(f(x_{n}))_{n \in \N}$ is the constant sequence $0$, which converges to $0$. Thus, we must have $f(x) = 0$, a contradiction. Hence, there exists $r > 0$ such that $f(y) \neq 0$ for all $y \in B(x,r)$.

    To show continuity of $g$, let $(y_{n})_{n \geq 1} \subseteq B(x,r)$ be a sequence such that $y_{n} \to y$ in $B(x,r)$. Since $f$ is continuous, we have $f(y_{n}) \to f(y)$ in $\R$. Since $f(y) \neq 0$, there exists a natural $N \in \N$ such that for every $n \geq N$, we have $f(y_{n}) \neq 0$. Thus, for every $n \geq N$, we have $g(y_{n}) = \frac{1}{f(y_{n})}$. Since the function $h : \R \setminus \{0\} \to \R$ defined by $h(t) = \frac{1}{t}$ is continuous, we have $g(y_{n}) = h(f(y_{n})) \to h(f(y)) = g(y)$ in $\R$, proving that $g$ is continuous.
\end{proof}

For a metric space $(X,d)$, we define $C(X,\R)$ to be the set of all continuous functions from $X$ to $\R$. It can be easily seen that $C(X,\R)$ is a vector space over $\R$ under pointwise addition and scalar multiplication. Moreover, if we define multiplication of functions pointwise, then $C(X,\R)$ is an algebra over $\R$. An example is any polynomial map $p:\R^{n} \to \R$; in such a case, $p \in C(\R^{n},\R)$.

If we denote $M_{2}(\R)$ to be the set of all $2 \times 2$ real matrices, then matrices can be viewed as points in $\R^{4}$, and the same Euclidean topology can be induced on $M_{2}(\R)$. Thus, we can consider continuous functions from $M_{2}(\R)$ to $\R$. An example is the determinant function $\det : M_{2}(\R) \to \R$ defined by $(a,b,c,d) \mapsto ad-bc$, which is a polynomial map and hence continuous. Since this is a continuous map, and $\R\setminus \{0\}$ is open in $\R$, the preimage $\det^{-1}(\R \setminus \{0\})$ is open in $M_{2}(\R)$. But $\det^{-1}(\R \setminus \{0\})$ is precisely the set of all invertible $2 \times 2$ real matrices $GL_{2}(\R)$. Thus, we have shown that $GL_{2}(\R)$ is an open subset of $M_{2}(\R)$. The set $SL_{2}(\R)$ of all $2 \times 2$ real matrices with determinant $1$, however, is closed in $M_{2}(\R)$, since it is the preimage of the closed set $\{1\}$ under the continuous determinant function.

\begin{example}
    Let us classify all $f:\R \to \R$ that are continuous and are also group homomorphisms. Since $f$ is continuous, we only need to determine $f$  $\R$ to $\R$ are of the form $f(x) = \lambda x$ for some fixedon the rationals. Let $\lambda \defeq f(1)$. Then, for every $m \in \Z$, $f(m) = m f(1) = m \lambda$. For every $n \in \N$, we have $n f(\frac{m}{n}) = f(n \cdot \frac{m}{n}) = f(m) = m \lambda$, so $f(\frac{m}{n}) = \frac{m}{n} \lambda$. Thus, for every rational number $q$, we have $f(q) = q \lambda$. Since the rationals are dense in $\R$ and $f$ is continuous, we must have $f(x) = x \lambda$ for all real numbers $x$. Thus, all continuous group homomorphisms from $\lambda \in \R$.
\end{example}


\section{Product Topology}
\noindent\textit{January 22nd.}

The concept of the product topology can be extended to arbitrary products. Let $\{(X_{n}, \tau_{n}) \mid n \geq 1\}$ be a collection of topological spaces. Consider the Cartesian product
\begin{align}
    \prod_{n=1}^{\infty} X_{n} = \{(x_{1}, x_{2}, \ldots) \mid x_{n} \in X_{n} \text{ for all } n \geq 1\}.
\end{align}
The product topology on $\prod_{n=1}^{\infty} X_{n}$ is the topology generated by the basis
\begin{align}
    \cB = \{U_{1} \times U_{2} \times \ldots \mid U_{n} \in \tau_{n} \text{ for all } n \geq 1,\; U_{i} \neq \emptyset \text{ for finitely many } i\}.
\end{align}
We require that $U_{i} \neq \emptyset$ for only finitely many $i$ to ensure that this is still a basis. This is known as the box topology. If this restriction were not in place, then the fact that countable intersections of open sets may not be open would imply that this is not a topology. A second option is to use the basis
\begin{align}
    \cB = \{U_{1} \times U_{2} \times \ldots \mid U_{n} \in \tau_{n} \text{ for all } n \geq 1,\; U_{i} = X_{i} \text{ for all but finitely many } i\}.
\end{align}
This is known as the product topology. Note that for finite products, both definitions coincide. In the infinite case, the product topology is more interesting than the box topology. The product topology also extends to a collection of topological spaces indexed by an arbitrary set, such as $\{(X_{\alpha}, \tau_{\alpha}) \mid \alpha \in \Lambda\}$ for some set $\Lambda$.

\begin{lemma}
    If $\{(X_{n}, \tau_{n}) \mid n \geq 1\}$ is a collection of Hausdorff topological spaces, then the product space $\prod_{n=1}^{\infty} X_{n}$ with the product topology is also Hausdorff.
\end{lemma}
\begin{proof}
    Let $x = (x_{1}, x_{2}, \ldots)$ and $y = (y_{1}, y_{2}, \ldots)$ be two distinct points in $\prod_{n=1}^{\infty} X_{n}$. Then, there exists some $k \geq 1$ such that $x_{k} \neq y_{k}$. Since $X_{k}$ is Hausdorff, there exist disjoint open sets $U_{k}, V_{k} \in \tau_{k}$ such that $x_{k} \in U_{k}$ and $y_{k} \in V_{k}$. Now, consider the open sets
    \begin{align}
        U = X_{1} \times X_{2} \times \ldots \times U_{k} \times X_{k+1} \times \ldots,
    \end{align}
    and
    \begin{align}
        V = X_{1} \times X_{2} \times \ldots \times V_{k} \times X_{k+1} \times \ldots.
    \end{align}
    Then, $U$ and $V$ are open in the product topology, and they are disjoint since $U_{k}$ and $V_{k}$ are disjoint. Moreover, $x \in U$ and $y \in V$, proving that $\prod_{n=1}^{\infty} X_{n}$ is Hausdorff.
\end{proof}

\begin{lemma}
    The function $f:Z \to \prod_{n=1}^{\infty} X_{n}$ defined by $f(z) = (f_{1}(z), f_{2}(z), \ldots)$ is continuous if and only if each component function $f_{n} : Z \to X_{n}$ is continuous for all $n \geq 1$. Here, $\prod_{n=1}^{\infty} X_{n}$ is equipped with the product topology.
\end{lemma}
\begin{proof}
    For the forward implication, let $n \geq 1$ be given. Consider an open set $U_{n} \in \tau_{n}$. Then, the set
    \begin{align}
        U = X_{1} \times X_{2} \times \ldots \times U_{n} \times X_{n+1} \times \ldots
    \end{align}
    is open in the product topology. Since $f$ is continuous, the preimage $f^{-1}(U)$ is open in $Z$. But,
    \begin{align}
        f^{-1}(U) = f_{n}^{-1}(U_{n}),
    \end{align}
    proving that $f_{n}$ is continuous.

    Conversely, suppose each component function $f_{n}$ is continuous for all $n \geq 1$. Let $U = U_{1} \times U_{2} \times \ldots$ be a basis element of the product topology, where $U_{n} \in \tau_{n}$ for all $n \geq 1$ and $U_{i} = X_{i}$ for all but finitely many $i$. Then,
    \begin{align}
        f^{-1}(U) = \bigcap_{n=1}^{\infty} f_{n}^{-1}(U_{n}) = \bigcap_{i=1}^{k} f_{i}^{-1}(U_{i}),
    \end{align}
    where $k$ is such that $U_{i} = X_{i}$ for all $i > k$. Since each $f_{i}$ is continuous, each $f_{i}^{-1}(U_{i})$ is open in $Z$. Thus, $f^{-1}(U)$ is open in $Z$, proving that $f$ is continuous.
\end{proof}

Note that the above lemma does not hold if the product topology is replaced by the box topology. Consider the function $f : \R \to \R^{\N}$ defined by $f(x) = (x,x,x,\ldots)$. Each component function $f_{n} : \R \to \R$ defined by $f_{n}(x) = x$ is continuous. However, $f$ is not continuous when $\R^{\N}$ is equipped with the box topology. To see this, consider the open set $(-1,1) \times (-\frac{1}{2}, \frac{1}{2}) \times (-\frac{1}{3}, \frac{1}{3}) \times \ldots$ in the box topology. The preimage under $f$ is $\{0\}$, which is not open in $\R$. Thus, $f$ is not continuous. However, the continuity of $f$ implies the continuity of each component function $f_{n}$, so the converse implication still holds for the box topology.
\\ \\
\textit{January 23rd.}

Consider $(\R,\bar{d})$ where $\bar{d}(x,y) = \min\{\abs{x-y}, 1\}$. This metric generates the usual topology on $\R$. Then, on $\R^{\N} = \prod_{n=1}^{\infty} \R$, we can consider the metric
\begin{align}
    \rho(x,y) = \sup_{n \geq 1} \bar{d}(x_{n}, y_{n}),
\end{align}
where $x = (x_{1}, x_{2}, \ldots)$ and $y = (y_{1}, y_{2}, \ldots)$. One can verify this is a metric. The topology generated by the metric space $(\R^{\N},\rho)$ is termed the \eax{uniform topology} on $\R^{\N}$. We now wish to determine the relationship between the uniform topology, product topology, and box topology on $\R^{\N}$. To determine this, we consider a few results.

\begin{lemma}
    $B_{\rho}(x,\varepsilon) \subsetneq \prod_{n=1}^{\infty} (x_{i}-\varepsilon,x_{i}+\varepsilon)$ for every $x = (x_{1}, x_{2}, \ldots) \in \R^{\N}$ and $(1 > )\varepsilon > 0$.
\end{lemma}
\begin{proof}
    Let $y = (y_{1}, y_{2}, \ldots) \in B_{\rho}(x,\varepsilon)$. Then, by definition of $\rho$, we have $\bar{d}(x_{n}, y_{n}) < \varepsilon$ for all $n \geq 1$. Since $\varepsilon < 1$, this implies that $\abs{x_{n} - y_{n}} < \varepsilon$ for all $n \geq 1$. Thus, $y \in \prod_{n=1}^{\infty} (x_{i}-\varepsilon,x_{i}+\varepsilon)$, proving that $B_{\rho}(x,\varepsilon) \subseteq \prod_{n=1}^{\infty} (x_{i}-\varepsilon,x_{i}+\varepsilon)$.

    To see that the inclusion is strict, consider the point $z = (z_{1}, z_{2}, \ldots)$ defined by $z_{i} = x_{i} - \varepsilon + \frac{1}{n+i}$ for all $i \geq 1$, for $n >> 1$ large enough. Then, $z \in \prod_{n=1}^{\infty} (x_{i}-\varepsilon,x_{i}+\varepsilon)$, but $\sup_{i \geq 1} \bar{d}(x_{i}, z_{i}) = \varepsilon$, so $z \notin B_{\rho}(x,\varepsilon)$. Thus, the inclusion is strict.
\end{proof}

\begin{lemma}
    On $\R^{\N}$, the product topology is strictly coarser than the uniform topology, which is strictly coarser than the box topology.
\end{lemma}
\begin{proof}
    We first show that $(x-\varepsilon,x+\varepsilon) \times \R \times \R \times \cdots$ is open in the uniform topology. Let $y = (y_{1}, y_{2}, \ldots) \in (x-\varepsilon,x+\varepsilon) \times \R \times \R \times \cdots$. Then, there exists a $\delta > 0$ such that $(y_{1}-\delta,y_{1}+\delta) \subseteq (x-\varepsilon,x+\varepsilon)$. Thus, $B(y,\delta) \subseteq (x-\varepsilon,x+\varepsilon) \times \R \times \R \times \cdots$, proving that this set is open in the uniform topology. Since basis elements of the product topology are finite intersections of such sets, they are also open in the uniform topology. Thus, the product topology is coarser than the uniform topology.
\end{proof}

\textit{January 27th.}

\begin{theorem}
    On $\R^{\N}$, with $D(x,y) \defeq \sup_{n \geq 1} \frac{\bar{d}(x_{n},y_{n})}{n}$ where $\bar{d}(x,y) = \min\{\abs{x-y}, 1\}$, the topology induced by $D$ is the product topology.
\end{theorem}

\begin{proof}
    For the forward implication, we wish to show that given an open ball $B_{D}(x,\varepsilon)$, and $y \in B_{D}(x,\varepsilon)$, that there exists a set $U$ open in the product topology such that $y \in U \subseteq B_{D}(x,\varepsilon)$. Choose $N \in \N$ such that $\frac{1}{N} < \frac{\varepsilon}{4}$. Now we claim that
    \begin{align}
        (x_{1} - \frac{\varepsilon}{3}, x_{1} + \frac{\varepsilon}{3}) \times (x_{2} - \frac{\varepsilon}{3}, x_{2} + \frac{\varepsilon}{3}) \times \cdots \times (x_{N+1} - \frac{\varepsilon}{3}, x_{N+1} + \frac{\varepsilon}{3}) \times \R \times \R \times \cdots \subseteq B_{D}(x,\varepsilon).
    \end{align}
    To this end, let $y = (y_{1}, y_{2}, \ldots)$ be in the left-hand side. Then for every $1 \leq i \leq N-1$, we have $\abs{y_{i}-x_{i}} < \frac{2\varepsilon}{3}$, so $\bar{d}(x_{i},y_{i})/i \leq \bar{d}(x_{i},y_{i}) < \frac{2\varepsilon}{3} < \varepsilon$. Moreover, for all $i \geq N$, $\bar{d}(x_{i},y_{i})/i \leq 1/i \leq 1/N < \frac{\varepsilon}{4} < \varepsilon$. Thus, $\sup_{i \geq 1} \bar{d}(x_{i},y_{i})/i = D(x,y) \leq \frac{2\varepsilon}{3} < \varepsilon$, proving the claim. Hence, we have found an open set in the product topology containing $x$ and contained in $U$.

    For the converse, let $U = \prod_{i \in \N} U_{i}$ be open in the product topology, where $U_{i}$ is open in $\R$ for all $i \geq 1$ and $U_{i} = \R$ for all but finitely many $i$. It is enough to containment for these sets, since they form a basis for the product topology. Thus, we wish to show that given any $x \in U$, there exists an $\varepsilon > 0$ such that $B_{D}(x,\varepsilon) \subseteq U$. Let $\alpha$ denote an arbitrary index such that $U_{\alpha} \neq \R$ (there are only finitely many such indices). Let $x \in U$. Then $x_{\alpha} \in U_{\alpha}$, so there exists $\varepsilon_{\alpha} > 0$ such that $(x_{\alpha} - \varepsilon_{\alpha}, x_{\alpha} + \varepsilon_{\alpha}) \subseteq U_{\alpha}$. Now, let $\varepsilon = \min_{\alpha} \frac{\varepsilon_{\alpha}}{\alpha}$, where the minimum is taken over all indices $\alpha$ such that $U_{\alpha} \neq \R$. We claim that this $\varepsilon$ works. To see this, let $y = (y_{1}, y_{2}, \ldots) \in B_{D}(x,\varepsilon)$. Then, for every index $\alpha$ such that $U_{\alpha} \neq \R$, we have
    \begin{align}
        \abs{y_{\alpha} - x_{\alpha}} \leq \bar{d}(x_{\alpha},y_{\alpha}) < \alpha \varepsilon \leq \varepsilon_{\alpha},
    \end{align}
    so $y_{\alpha} \in (x_{\alpha} - \varepsilon_{\alpha}, x_{\alpha} + \varepsilon_{\alpha}) \subseteq U_{\alpha}$. For all other indices $\beta$ such that $U_{\beta} = \R$, we have $y_{\beta} \in U_{\beta}$ trivially. Thus, $y \in U$, proving that $B_{D}(x,\varepsilon) \subseteq U$.
\end{proof}

The box topology, however, is not metrizable. 

\begin{lemma}
    The box topology on $\R^{\N}$ is not metrizable.
\end{lemma}
To show this, recall that in a metric space if $A$ is a subset of a metric space $(X,d)$, and $(a_{n})_{n \geq 1}$ is a sequence in $A$ that converges to some $x \in X$, then $x \in \overline{A}$. The converse also holds in metric spaces. 

\begin{proof}
    Let $A = \{(x_{1}, x_{2}, \ldots) \mid x_{i} > 0 \text{ for all } i \geq 1\} \subseteq \R^{\N}$ with the box topology. Note that $A$ is open in the box topology. Now consider the point $x = (0,0,0,\ldots) \in \R^{\N}$. We claim that $x \in \overline{A}$. To see this, let $U = U_{1} \times U_{2} \times \ldots$ be an open neighbourhood of $x$ in the box topology. Then, for each $i \geq 1$, there exists $\varepsilon_{i} > 0$ such that $(-\varepsilon_{i}, \varepsilon_{i}) \subseteq U_{i}$. Now, consider the point $y = (y_{1}, y_{2}, \ldots)$ defined by $y_{i} = \frac{\varepsilon_{i}}{2} > 0$ for all $i \geq 1$. Then, $y \in A \cap U$, proving that every open neighbourhood of $x$ intersects $A$. Thus, $x \in \overline{A}$.
    However, there does not exist a sequence $(a_{n})_{n \geq 1} \subseteq A$ such that $a_{n} \to x$ in the box topology. To see this, suppose such a sequence exists. Then, for each $i \geq 1$, consider the open neighbourhood $U^{(i)} = U_{1}^{(i)} \times U_{2}^{(i)} \times \ldots$ of $x$ defined by
    \begin{align}
        U_{j}^{(i)} = \begin{cases}
            (-\frac{1}{i}, \frac{1}{i}) & j = i, \\
            \R & j \neq i.
        \end{cases}
    \end{align}
    Since $a_{n} \to x$ in the box topology, there exists a natural $N_{i} \in \N$ such that for every $n \geq N_{i}$, we have $a_{n} \in U^{(i)}$. In particular, this implies that for every $n \geq N_{i}$, the $i$-th coordinate of $a_{n}$ satisfies $\abs{(a_{n})_{i}} < \frac{1}{i}$. Now, consider the sequence of natural numbers $(N_{i})_{i \geq 1}$. Since this is an increasing sequence, we have $N_{i} \leq N_{j}$ for all $j \geq i$. Thus, for every $j \geq i$, we have $\abs{(a_{N_{j}})_{i}} < \frac{1}{i}$. In particular, this implies that $(a_{N_{j}})_{i} \geq 0$ for all $j \geq i$, since $a_{N_{j}} \in A$. Thus, for each fixed $i \geq 1$, the sequence $((a_{N_{j}})_{i})_{j \geq i}$ is a sequence of non-negative real numbers that converges to $0$. However, this does not guarantee that the entire sequence $(a_{n})_{n \geq 1}$ converges to $x$ in the box topology, since the convergence must hold uniformly across all coordinates. This contradicts our assumption that such a sequence exists. Hence, there does not exist a sequence $(a_{n})_{n \geq 1} \subseteq A$ such that $a_{n} \to x$ in the box topology, proving that the box topology is not metrizable.
\end{proof}


\section{Connectedness}

The \eax{distance from a set} $A$ of a point $x$ is defined as
\begin{align}
    d_{A}(x) \defeq \inf \{d(x,a) \mid a \in A\}.
\end{align}
Note that $d_{A}(x) = 0$ if and only if $x \in \overline{A}$. Moreover, the map $\varphi : X \to \R$ defined by $\varphi(x) = d_{A}(x)$ is continuous.

\begin{lemma}
    Let $(X,d)$ be a metric space, and $A,B \subseteq X$ be disjoint closed sets. Then there exists a continuous map $f:X \to [0,1]$ such that $f(a) = 0$ for all $a \in A$ and $f(b) = 1$ for all $b \in B$.
\end{lemma}

\begin{definition}
    For a topological space $(X,\tau)$, a \eax{separation} of $X$ is a pair of disjoint non-empty open sets $U,V \in \tau$ such that $X = U \cup V$. If no such separation exists, then $X$ is said to be \eax{connected}.
\end{definition}
Note that $X$ is connected if and only if the only non-empty clopen subset of $X$ is $X$ itself. For example, $\Q$ is not connected in $\R$ since $(-\infty, \sqrt{2}) \cap \Q$, $ (\sqrt{2}, \infty) \cap \Q$ is a separation of $\Q$.

\begin{lemma}
    Suppose $C$ and $D$ form a separation of a topological space $X$. If $Y$ is a connected subspace of $X$, then either $Y \subseteq C$ or $Y \subseteq D$.
\end{lemma}
\begin{proof}
    Suppose not. Then, there exist points $y_{1}, y_{2} \in Y$ such that $y_{1} \in C$ and $y_{2} \in D$. Since $C$ and $D$ are open in $X$, the sets $C \cap Y$ and $D \cap Y$ are open in the subspace topology on $Y$. Moreover, they are disjoint and non-empty, and $Y = (C \cap Y) \cup (D \cap Y)$. Thus, $C \cap Y$ and $D \cap Y$ form a separation of $Y$, contradicting the connectedness of $Y$. Hence, either $Y \subseteq C$ or $Y \subseteq D$.
\end{proof}

\begin{theorem}
    Suppose $X_{\alpha}$ are connected subspaces of a topological space $X$ for all $\alpha$ in some index set $\Lambda$. If $\bigcap_{\alpha \in \Lambda} X_{\alpha} \neq \emptyset$, then $\bigcup_{\alpha \in \Lambda} X_{\alpha}$ is connected.
\end{theorem}
\begin{proof}
    Let $x \in \bigcap_{\alpha} X_{\alpha}$. Suppose, for contradicion, that there exists a non-empty set $U \subseteq \bigcup_{\alpha} X_{\alpha}$ that is clopen in the subspace topology. That is, $U \cap X_{\alpha}$ is open and closed in $X_{\alpha}$ for all $\alpha \in \Lambda$. Since $X_{\alpha}$ is connected, we must have either $U \cap X_{\alpha} = \emptyset$ or $U \cap X_{\alpha} = X_{\alpha}$ for all $\alpha \in \Lambda$. But since $x \in \bigcap_{\alpha} X_{\alpha}$, we must have $x \in U$ or $x \notin U$. Thus, either $U = \emptyset$ or $U = \bigcup_{\alpha} X_{\alpha}$, proving that $\bigcup_{\alpha} X_{\alpha}$ is connected.
\end{proof}

\begin{lemma}
    Let $f:X \to Y$ be a continuous map. If $X$ is connected, then $f(X)$ is also connected.
\end{lemma}
\begin{proof}
    Suppose not. Then, there exist disjoint non-empty open sets $U,V \subseteq Y$ such that $f(X) = U \cup V$. Since $f$ is continuous, the preimages $f^{-1}(U)$ and $f^{-1}(V)$ are open in $X$. Moreover, they are disjoint and non-empty, and $X = f^{-1}(U) \cup f^{-1}(V)$. Thus, $f^{-1}(U)$ and $f^{-1}(V)$ form a separation of $X$, contradicting the connectedness of $X$. Hence, $f(X)$ is connected.
\end{proof}

\begin{definition}
    Two topological spaces $X$ and $Y$ are \eax{homeomorphic} if there exists a bijective continuous map $f:X \to Y$ such that the inverse map $f^{-1} : Y \to X$ is also continuous. Such a map $f$ is called a \eax{homeomorphism}.
\end{definition}

\textit{January 30th.}

\begin{theorem}
    If $X_{i}$ are connected topological spaces for all $1 \leq i \leq n$, then the product space $\prod_{i=1}^{n} X_{i}$ is also connected.
\end{theorem}
\begin{proof}
    Note that $X_{1} \times \{b\}$ is connected for $b \in X_{2}$, since it is homeomorphic to $X_{1}$. Now, let $a \in X_{1}$ be arbitrary. Then, the set $\{a\} \times X_{2}$ is connected. Thus, the union $X_{1} \times X_{2}$ is connected, since $(X_{1} \times \{b\}) \cap (\{a\} \times X_{2}) = \{(a,b)\} \neq \emptyset$. Repeating this argument inductively, we have that $\prod_{i=1}^{n} X_{i}$ is connected.
\end{proof}

\begin{theorem}
    If $A \subseteq X$ is connected and $A \subseteq B \subseteq \overline{A}$, then $B$ is also connected.
\end{theorem}
\begin{proof}
    Assume that $B$ is disconnected. That is, there exists a clopen set $U$ in $B$ such that $B = U \cup U^{c}$. Without the loss of generality, suppose $\emptyset \neq U \cap A$ is clopen in $A$. Then $U \cap A = A$ or $A \subseteq U$. But then $\overline{A} \subseteq \overline{U} = U$, so $B \subseteq U$, a contradiction. Hence, $B$ is connected.
\end{proof}

We will now show that $\R^{\N}$ with the product topology is connected, but not with the box topology. With the product topology equipped, let $A_{n} = \{(x_{1},\ldots,x_{n},0,\ldots) \mid x_{i} \in \R\}$. Then $\R^{\N} = \bigcup_{n=1}^{\infty} A_{n}$, and each $A_{n}$ is homeomorphic to $\R^{n}$, which is connected. Moreover, $A_{n} \subseteq A_{n+1}$ for all $n \geq 1$, so by the previous theorem, $\R^{\N}$ with the product topology is connected. For the box topology, let $\R^{\N} = B \cup U$, where $B$ is the set of bounded sequences, and $U$ is the set of unbounded sequences. To show that $B$ is open, let $x = (x_{1}, x_{2}, \ldots) \in B$. Then there exists $K > 0$ such that $\abs{x_{i}} < K$ for all $i \geq 1$. Finally, $x_{i} \in (x_{1}-1, x_{1}+1) \times (x_{2}-1, x_{2}+1) \times \ldots$ is an open neighbourhood of $x$ contained in $B$, proving that $B$ is open. A similar argument shows that $U$ is open. Since both $B$ and $U$ are non-empty, they form a separation of $\R^{\N}$ with the box topology, proving that it is not connected.

\begin{definition}
    Let $a \neq b \in X$. We say that $\gamma$ is a \eax{path} between $a$ and $b$ if $\gamma : [0,1] \to X$ is continuous, with $\gamma(0) = a$ and $\gamma(1) = b$. If such a path exists, then $a$ and $b$ are said to be path connected. A topological space $X$ is \eax{path connected} if every pair of points in $X$ are path connected.
\end{definition}

\begin{lemma}
    Supose there exists a $a_{0} \in X$ such that $a_{0}$ is path connected to every point in $X$. Then, $X$ is path connected.
\end{lemma}
\begin{proof}
    Let $a,b \in X$ be arbitrary. By assumption, there exist paths $\gamma_{1} : [0,1] \to X$ and $\gamma_{2} : [0,1] \to X$ such that $\gamma_{1}(0) = a_{0}$, $\gamma_{1}(1) = a$, $\gamma_{2}(0) = a_{0}$, and $\gamma_{2}(1) = b$. Now, define the map $\gamma : [0,1] \to X$ by
    \begin{align}
        \gamma(t) = \begin{cases}
            \gamma_{1}(2t) & 0 \leq t \leq \frac{1}{2}, \\
            \gamma_{2}(2t-1) & \frac{1}{2} < t \leq 1.
        \end{cases}
    \end{align}
    Then, $\gamma$ is continuous, and $\gamma(0) = a$, $\gamma(1) = b$. Thus, $a$ and $b$ are path connected. Since $a,b$ were arbitrary, $X$ is path connected.
\end{proof}

\begin{theorem}
    If $X$ is path connected, then $X$ is connected.
\end{theorem}
\begin{proof}
    Suppose $X$ is not connected. Then, there exists a separation $U,V$ of $X$ such that $U \cap V = \emptyset$. Let $a \in U$ and $b \in V$. Since $X$ is path connected, there exists a path $\gamma : [0,1] \to X$ such that $\gamma(0) = a$ and $\gamma(1) = b$. Consider the preimages $\gamma^{-1}(U)$ and $\gamma^{-1}(V)$. These sets are open in $[0,1]$, disjoint, non-empty, and their union is $[0,1]$. Thus, they form a separation of $[0,1]$, contradicting the connectedness of $[0,1]$. Hence, $X$ is connected.
\end{proof}

\noindent \textit{February 3rd.}

Note that in the above proof we implicitly assumed the fact that $[0,1]$ is connected. To show that it is connected, we require the following lemma which is left as an exercise.

\begin{lemma}
    A topological space $X$ is connected if and only if every continuous map $f:X \to \{\pm1\}$ is constant.
\end{lemma}

\begin{theorem}
    The interval $[0,1]$ is connected.
\end{theorem}
\begin{proof}
    Suppose $f:[0,1] \to \{\pm1\}$ is continuous, and not constant. Then there exist $a,b \in [0,1]$ such that $f(a) = 1$ and $f(b) = -1$. Without loss of generality, assume $a < b$. By the intermediate value theorem, there exists some $c \in (a,b)$ such that $f(c) = 0$, contradicting the fact that the codomain of $f$ is $\{\pm1\}$. Thus, every continuous map $f:[0,1] \to \{\pm1\}$ is constant, proving that $[0,1]$ is connected.
\end{proof}

\begin{proof}[Alternate proof]
    Suppose $U \subseteq [0,1]$ is non-empty, open, and closed in the subspace topology. Without loss of generality, let $0 \in U$ (otherwise, take $V = U^{c}$). Now define $E = \{x \in [0,1] \mid [0,x] \subseteq U\}$. Since $0 \in U$, we have $0 \in E$, so $E$ is non-empty. There now exists a natural $N \in \N$ such that $[0,\frac{1}{n}] \subseteq U$ for all $n \geq N$, so $\frac{1}{N} \in E$. Thus, $E$ is bounded above by $1$, so let $c = \sup E$. We claim that $c = 1$. Suppose not. Since $U$ is closed in $[0,1]$, we have $c \in U$. Thus, there exists some $\varepsilon > 0$ such that $(c-\varepsilon, c+\varepsilon) \cap [0,1] \subseteq U$. But since $c = \sup E$, there exists some $x \in E$ such that $c - \varepsilon < x \leq c$. Then, $[0,x] \subseteq U$, and $(c-\varepsilon, c+\varepsilon) \cap [0,1] \subseteq U$ implies that $[0, c + \frac{\varepsilon}{2}] \subseteq U$, contradicting the fact that $c$ is an upper bound for $E$. Thus, $c = 1$, so $[0,1] \subseteq U$, proving that $U = [0,1]$. Hence, $[0,1]$ is connected.
\end{proof}

\begin{example}
    We can use to show that $GL_{n}(\R)$ is not connected. Note that the map $\det : GL_{n}(\R) \to \R \setminus \{0\}$ is continuous. Since $\R \setminus \{0\}$ is not connected, $GL_{n}(\R)$ is also not connected. Similarly, the hypersphere $S^{n} = \{x \in \R^{n+1} \mid \norm{x} = 1\}$ is connected for all $n \geq 1$, since the map $f : \R^{n+1} \setminus \{0\} \to S^{n}$ defined by $f(x) = \frac{x}{\norm{x}}$ is continuous, and $\R^{n+1} \setminus \{0\}$ is connected for all $n \geq 1$.
\end{example}

\begin{example}
    We show that $\R$ and $\R^{2}$ cannot be homeomorphic. Suppose there exists a homeomorphism $f : \R \to \R^{2}$. Let $a \in \R$ be arbitrary, and consider the set $\R^{2} \setminus \{f(a)\}$. Note that this set is connected, since it is homeomorphic to $\R^{2} \setminus \{0\}$, which is connected. However, the set $\R \setminus \{a\}$ is not connected, since it can be separated into $(-\infty,a)$ and $(a,\infty)$. But since $f$ is a homeomorphism, the image of $\R \setminus \{a\}$ under $f$ must also be disconnected, a contradiction. Thus, no such homeomorphism exists, proving that $\R$ and $\R^{2}$ are not homeomorphic.
\end{example}

\subsection{Components}
On a topological space $X$, define an equivalence relation $\sim$ on $X$ by $x \sim y$ if and only if there exists a connected subspace of $X$ containing both $x$ and $y$. The equivalence classes under this relation are called the \eax{component}s of $X$.

\begin{theorem}
    The components of $X$ are connected, disjoint subspaces such that their union is $X$. Any connected subspace of $X$ intersects only one component.
\end{theorem}
\begin{proof}
    Let $C$ be a component of $X$. Let $x_{0} \in C$ be arbitrary. Then, by definition of the equivalence relation, for every $x \in C$, there exists a connected subspace $V_{x}$ of $X$ such that $x_{0}, x \in V_{x}$. Thus, we have
    \begin{align}
        C = \bigcup_{x \in C} V_{x},
    \end{align}
    and since each $V_{x}$ is connected and they all share the point $x_{0}$, by a previous theorem, $C$ is connected. The components are disjoint simply by the definition of an equivalence relation, and their union is $X$ since every point in $X$ is contained in some component. Finally, suppose $A \subseteq X$ is connected, and intersects two components $C_{1}$ and $C_{2}$. Then, there exist points $a_{1} \in A \cap C_{1}$ and $a_{2} \in A \cap C_{2}$. Since $A$ is connected, there exists a connected subspace of $X$ containing both $a_{1}$ and $a_{2}$. Thus, by definition of the equivalence relation, we must have $C_{1} = C_{2}$, proving that any connected subspace of $X$ intersects only one component.
\end{proof}

Thus, the components may even be called the connected components of $X$. 

\begin{theorem}
    Let $U \subseteq \R^{n}$ be an open connected set. Then, $U$ is path connected.
\end{theorem}
\begin{proof}
    Fix some $x_{0} \in U$, and let $A$ be the set of all points in $U$ that are path connected to $x_{0}$. We wish to show that $A = U$. Note that $A$ is non-empty since $x_{0} \in A$. To show that $A$ is open, let $x \in A$ be arbitrary. Since $U$ is open, there exists some $\varepsilon > 0$ such that $B(x,\varepsilon) \subseteq U$. Now, for any $y \in B(x,\varepsilon)$, the line segment from $x$ to $y$ lies entirely within $B(x,\varepsilon)$, so there exists a path from $x$ to $y$. Since there also exists a path from $x_{0}$ to $x$, by concatenating these two paths, we have a path from $x_{0}$ to $y$. Thus, $y \in A$, proving that $B(x,\varepsilon) \subseteq A$. Hence, $A$ is open. To show that $A$ is closed, we look at $U \setminus A$. Let $x \in U \setminus A$ be arbitrary; there is no path from $x_{0}$ to $x$. Since $U$ is open, there exists some $\delta > 0$ such that $B(x,\delta) \subseteq U$. Now, for any $y \in B(x,\delta)$, the line segment from $x$ to $y$ lies entirely within $B(x,\delta)$, so there exists a path from $x$ to $y$. If there existed a path from $x_{0}$ to $y$, then by concatenating this path with the path from $x$ to $y$, we would have a path from $x_{0}$ to $x$, contradicting our assumption. Thus, there is no path from $x_{0}$ to $y$, so $y \in U \setminus A$. Hence, $B(x,\delta) \subseteq U \setminus A$, proving that $U \setminus A$ is open. Since $U$ is connected, and $A$ is non-empty, open, and closed in $U$, we must have $A = U$. Thus, every point in $U$ is path connected to $x_{0}$, so $U$ is path connected.
\end{proof}

More generally, we have the following result.

\begin{theorem}
    Let $X$ be a topological space that is connected. If each point $x \in X$ has an open neighbourhood that is path connected, then $X$ is path connected.
\end{theorem}

From here, we say that $X$ is locally connected at $x_{0}$ if every open neighbourhood of $x_{0}$ contains a connected open neighbourhood of $x_{0}$. If $X$ is locally connected at every point $x \in X$, then $X$ is said to be \eax{locally connected}.


\section{Compactness}

Recall that $\{A_{\alpha}\}_{\alpha \in \Lambda}$ is an open \eax{covering} of $X$ if $\bigcup_{\alpha \in \Lambda} A_{\alpha} \supseteq X$. A subcover is a subset of the covering that still covers $X$.

\begin{definition}
    A topological space $X$ is \eax{compact} if every open covering of $X$ has a finite subcovering.
\end{definition}

\begin{theorem}
    Every compact subset of a Hausdorff space is closed.
\end{theorem}
\begin{proof}
    Let $K \subseteq X$ be compact. We wish to show that $K^{c}$ is open. Let $x_{0} \in X \setminus K$. For each $y \in K$, since $X$ is Hausdorff, there exist disjoint open neighbourhoods $U_{y}$ of $x_{0}$ and $V_{y}$ of $y$. Then $\{V_{y}\}_{y \in K}$ is an open covering of $K$, so there exists a finite subcovering $\{V_{y_{i}}\}_{i=1}^{n}$. Now, let $U = \bigcap_{i=1}^{n} U_{y_{i}}$. Then, $U$ is an open neighbourhood of $x_{0}$, and for every $i$, we have $U \cap V_{y_{i}} = \emptyset$. Thus, $U \cap K = \emptyset$, proving that $x_{0} \in K^{c}$ has an open neighbourhood contained in $K^{c}$. Since $x_{0}$ was arbitrary, $K^{c}$ is open, so $K$ is closed.
\end{proof}

\begin{lemma}
    Let $F \subseteq K$ where $F$ is closed and $K$ is compact. Then, $F$ is also compact.
\end{lemma}
\begin{proof}
    Let $\{U_{\alpha}\}_{\alpha \in \Lambda}$ be an open covering of $F$. Then, $\{U_{\alpha}\}_{\alpha \in \Lambda} \cup \{K \setminus F\}$ is an open covering of $K$, so there exists a finite subcovering $\{U_{\alpha_{i}}\}_{i=1}^{n} \cup \{K \setminus F\}$. Since $F$ is not covered by $K \setminus F$, we have that $\{U_{\alpha_{i}}\}_{i=1}^{n}$ is a finite subcovering of $F$. Thus, $F$ is compact.
\end{proof}

\begin{theorem}
    $[a,b]$ is compact for all $a < b \in \R$.
\end{theorem}
\begin{proof}
    Let $\{U_{\alpha}\}_{\alpha \in \Lambda}$ be an open covering of $[a,b]$. Suppose it is not compact. Choose $a \leq a_{1} < b_{1} \leq b$ such that $I_{1} = [a_{1},b_{1}]$ has no finite subcovering, and $\abs{b_{1}-a_{1}} \leq \frac{1}{2}\abs{b-a}$. Now, choose $a_{1} \leq a_{2} < b_{2} \leq b_{1}$ such that $I_{2} = [a_{2},b_{2}]$ has no finite subcovering, and $\abs{b_{2}-a_{2}} \leq \frac{1}{2}\abs{b_{1}-a_{1}}$. Continuing this process, we have a sequence of nested closed intervals $I_{n} = [a_{n}, b_{n}]$ such that $I_{n}$ has no finite subcovering for all $n \geq 1$, and $\abs{b_{n}-a_{n}} \leq \frac{1}{2^{n-1}}\abs{b-a}$ for all $n \geq 1$. By the nested interval property, there exists some $c \in [a,b]$ such that $\sup a_{k} = c = \inf b_{k}$. Then $c \in U_{\alpha_{0}}$ for some $\alpha_{0} \in \Lambda$, since $\{U_{\alpha}\}_{\alpha \in \Lambda}$ is an open covering of $[a,b]$. Thus, there exists some $\varepsilon > 0$ such that $(c-\varepsilon, c+\varepsilon) \subseteq U_{\alpha_{0}}$. However, there exists some $N \in \N$ such that for all $n \geq N$, we have $\abs{b_{n}-a_{n}} < \varepsilon$, so $I_{N} \subseteq (c-\varepsilon, c+\varepsilon) \subseteq U_{\alpha_{0}}$. This contradicts the fact that $I_{N}$ has no finite subcovering. Hence, $[a,b]$ is compact.
\end{proof}

\begin{lemma}
    Let $f:X \to Y$ be a continuous map. If $X$ is compact, then $f(X)$ is also compact.
\end{lemma}
\begin{proof}
    Let $f(K) \subseteq \bigcup_{\alpha \in \Lambda} U_{\alpha}$ be an open covering of $f(K)$. Then, $\{f^{-1}(U_{\alpha})\}_{\alpha \in \Lambda}$ is an open covering of $K$, so there exists a finite subcovering $\{f^{-1}(U_{\alpha_{i}})\}_{i=1}^{n}$. Thus, $\{U_{\alpha_{i}}\}_{i=1}^{n}$ is a finite subcovering of $f(K)$, proving that $f(K)$ is compact.
\end{proof}

\begin{theorem}
    Let $f:X \to Y$ be a continuous bijection. If $X$ is compact and $Y$ is Hausdorff, then $f$ is a homeomorphism.
\end{theorem}
\begin{proof}
    We wish to show that the inverse map $f^{-1} : Y \to X$ is continuous. Let $V \subseteq X$ be closed. Then, since $X$ is compact, $V$ is also compact. Thus, by the previous lemma, $f(V)$ is compact. Since $Y$ is Hausdorff, $f(V)$ is closed. Hence, the preimage of every closed set under $f^{-1}$ is closed, proving that $f^{-1}$ is continuous. Thus, $f$ is a homeomorphism.
\end{proof}

\begin{theorem}
    If $X$ and $Y$ are compact topological spaces, then the product space $X \times Y$ is also compact.
\end{theorem}
\begin{proof}
    Note that for $x_{0} \in X$, $x_{0} \times Y$ is homeomorphic to $Y$, so it is compact. We need to show that if $N$ is an open set such that $x_{0} \times Y \subset N \subset X \times Y$, then there exists a neighbourhood $W$ of $x_{0}$ in $X$ such that $W \times Y \subseteq N$. Let $X \times Y = \bigcup_{\alpha} A_{\alpha}$ be an open covering of $X \times Y$. By compactness of $x_{0} \times Y$, there exists a finite subcovering $\bigcup_{i \in \Lambda_{x_{0}}} A_{\alpha_{i}}$ where $\Lambda_{x_{0}}$ is a finite index set. Then $x_{0} \times Y \subseteq W_{x_{0}} \times Y \subseteq \bigcup_{i \in \Lambda_{x_{0}}} A_{\alpha_{i}}$, where $W_{x_{0}} = \bigcap_{i \in \Lambda_{x_{0}}} \pi_{X}(A_{\alpha_{i}})$ is an open neighbourhood of $x_{0}$ in $X$. Now, for each $x \in X$, we can repeat this process to obtain an open neighbourhood $W_{x}$ of $x$ in $X$ such that $W_{x} \times Y$ is covered by a finite number of sets from the open covering. Then, $\{W_{x}\}_{x \in X}$ is an open covering of $X$, so by compactness of $X$, there exists a finite subcovering $\{W_{x_{j}}\}_{j=1}^{m}$. Thus, we have $X \times Y = \bigcup_{j=1}^{m} (W_{x_{j}} \times Y)$, and each $W_{x_{j}} \times Y$ is covered by a finite number of sets from the original open covering. Hence, $X \times Y$ has a finite subcovering, proving that it is compact.
\end{proof}